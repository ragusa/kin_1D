%%%%%%%%%%%%%%%%%%%%%%%%%%%%%%%%%%%%%%%%%%%%%%%%%%%%%%%%%%%%%
\documentclass[12pt]{scrartcl} 
%########################### Preferences #################################

\usepackage{vmargin}
\usepackage{color}
\usepackage{amsthm}
\usepackage{amssymb}
\usepackage{amsmath}
\usepackage{amsfonts}
\usepackage{amstext}
\usepackage{amsbsy}
%\usepackage{mathbbol}
\usepackage{graphicx} 
\usepackage{verbatim}
\usepackage{csvsimple} 
\usepackage{subcaption}
\usepackage{hyperref}
\usepackage{fancyhdr}
\usepackage{multirow}
\usepackage{tikz}
\usepackage{listings}
\usepackage[numbers,sort&compress]{natbib}[2010/09/13]

%  new definitions
\renewcommand{\div}{\vec{\nabla}\! \cdot \!}
\newcommand{\grad}{\vec{\nabla}}
\newcommand{\norm}[1]{\left\lVert#1\right\rVert_{L^2}}
\newcommand{\rattlesnake}{Rattlesnake }
% extra space
\newcommand{\qq}{\quad\quad}
% common reference commands`
\newcommand{\eqt}[1]{Equation~\ref{#1}}                     % equation
\newcommand{\fig}[1]{Figure~\ref{#1}}                      % figure
\newcommand{\tbl}[1]{Table~\ref{#1}}                     % table
\newcommand{\sct}[1]{Section~\ref{#1}}                   % section
\newcommand{\app}[1]{Appendix~\ref{#1}}                   % appendix

\newcommand{\bs}[1]{\mathbf{#1}}
\newcommand{\dd}{\mathrm{d}}
\newcommand{\keff}{k_\textit{eff}}

\newcommand{\be}{\begin{equation}}
\newcommand{\ee}{\end{equation}}
\newcommand{\vn}{\vec{n}}
\newcommand{\vel}{\vec{\mathrm{v}}}
\newcommand{\adj}{\Phi^\dagger_0}
\newcommand{\tcr}[1]{\textcolor{red}{#1}}
\newcommand{\tcb}[1]{\textcolor{blue}{#1}}

% tikz stuff
\usetikzlibrary{shapes,arrows,positioning}
\pgfdeclarelayer{background}
\pgfdeclarelayer{foreground}
\pgfsetlayers{background,main,foreground}
\tikzstyle{greenblock}=[rectangle, rounded corners, draw, align=center, top color=white, bottom color=green!20, ultra thick, minimum width=60mm, minimum height=15mm,]
\tikzstyle{blueblock}=[rectangle, rounded corners, draw, align=center, top color=white, bottom color=blue!20, ultra thick, minimum width=60mm, minimum height=15mm]
\tikzstyle{reddiamond}=[diamond, draw, align=center, top color=white, bottom color=red!20, ultra thick, minimum width=60mm, aspect=2]
\newcommand{\tikzback}[5]{
 \begin{pgfonlayer}{background}
  \path (#1.west |- #2.north)+(-0.5,0.75) node (a1) {};
  \path (#3.east |- #4.south)+(+0.5,-0.25) node (a2) {};
  \path[fill=yellow!10,rounded corners, draw=black!100, dashed] (a1) rectangle (a2);
   \path (#3.east |- #2.north)+(0,1.25)--(#1.west |- #2.north) node[midway] (#5-n) {};
   \path (#3.east |- #2.south)+(0,-0.35)--(#1.west |- #2.south) node[midway] (#5-s) {};
   \path (#1.west |- #2.north)+(-0.75,0)--(#1.west |- #4.south) node[midway] (#5-w) {};
 \end{pgfonlayer}}
 
% input stuff
\lstset{%
%  language=C++,
  showstringspaces=false,
% basicstyle=\scriptsize\ttfamily,
  basicstyle=\footnotesize\ttfamily,
  commentstyle=\color{inl@green},
%  keywordstyle=\bfseries,
%  escapeinside=$$
%  framesep=0pt,
%  rulesep=0pt,
  frame=single
}
% Make nice big tildes inside lstlistings instead of
% tiny ones.
\lstset{
    literate={~} {$\sim$}{1}
}

% ********* Caption Layout ************
%\usepackage{ccaption} % allows special formating of the captions
%\captionnamefont{\bf\footnotesize\sffamily} % defines the font of the caption name (e.g. Figure: or Table:)
%\captiontitlefont{\footnotesize\sffamily} % defines the font of the caption text (same as above, but not bold)
%\setlength{\abovecaptionskip}{0mm} %lowers the distance of captions to the figure
\setlength\parindent{0pt}
\setlength{\oddsidemargin}{1in}
\setlength{\evensidemargin}{1in}
\setlength{\textwidth}{6.5in}
% ********* Header and Footer **********
% This is something to play with forever. I use here the advanced settings of the KOMA script

\title{Uncollided Flux Techniques for Discrete-Ordinate Radiation Transport Solutions in \rattlesnake}
\subtitle{Uncollided Flux Treatment \rattlesnake}
\author{ \normalsize
  \textbf{Jean C. Ragusa$^\dagger$, Derek Gaston$^\star$, Mark D. DeHart$^\star$} \\
 \normalsize $^\dagger$Texas A\&M University, College Station, TX, USA\\
 \normalsize $^\star$Idaho National Laboratory, Idaho Falls, ID, USA\\
 \normalsize \href{jean.ragusa@tamu.edu}{jean.ragusa@tamu.edu}, \href{derek.gaston@inl.gov}{derek.gaston@inl.gov}, \href{mark.dehart@inl.gov}{mark.dehart@inl.gov}
}

\hypersetup{
  colorlinks=true,
  urlcolor=blue,
}

\pagestyle{fancy}
\fancyhf{}
\lhead{INL/EXT-16-xxxxx}
\rfoot{DRAFT}
\cfoot{\thepage}


%%%%%%%%%%%%%%%%%%%%%%%%%%%%%%%% boxes
\usepackage{color}
\definecolor{myblue}{rgb}{.8, .8, 1}
\usepackage{empheq}

\newlength\mytemplen
\newsavebox\mytempbox

\makeatletter
\newcommand\mybluebox{%
    \@ifnextchar[%]
       {\@mybluebox}%
       {\@mybluebox[0pt]}}

\def\@mybluebox[#1]{%
    \@ifnextchar[%]
       {\@@mybluebox[#1]}%
       {\@@mybluebox[#1][0pt]}}

\def\@@mybluebox[#1][#2]#3{
    \sbox\mytempbox{#3}%
    \mytemplen\ht\mytempbox
    \advance\mytemplen #1\relax
    \ht\mytempbox\mytemplen
    \mytemplen\dp\mytempbox
    \advance\mytemplen #2\relax
    \dp\mytempbox\mytemplen
    \colorbox{myblue}{\hspace{1em}\usebox{\mytempbox}\hspace{1em}}}

\makeatother

%################ End Preferences, Begin Document #####################

%\pagestyle{plain} % on headers or footers on the first page

%%%%%%%%%%%%%%%%%%%%%%%%%%%%%%%%%%%%%%%%%%%%%%%%%%%%%%%%%%%%%%%%%%%%%%%%%%%%%
%%%%%%%%%%%%%%%%%%%%%%%%%%%%%%%%%%%%%%%%%%%%%%%%%%%%%%%%%%%%%%%%%%%%%%%%%%%%%
% set relative directory for figures. can be changed at any time later with \renewcommand
\newcommand{\FiguresDir}{./figs}

%%%%%%%%%%%%%%%%%%%%%%%%%%%%%%%%%%%%%%%%%%%%%%%%%%%%%%%%%%%%%%%%%%%%%%%%%%%%%
%%%%%%%%%%%%%%%%%%%%%%%%%%%%%%%%%%%%%%%%%%%%%%%%%%%%%%%%%%%%%%%%%%%%%%%%%%%%%
\begin{document}
%%%%%%%%%%%%%%%%%%%%%%%%%%%%%%%%%%%%%%%%%%%%%%%%%%%%%%%%%%%%%%%%%%%%%%%%%%%%%
%%%%%%%%%%%%%%%%%%%%%%%%%%%%%%%%%%%%%%%%%%%%%%%%%%%%%%%%%%%%%%%%%%%%%%%%%%%%%
\maketitle
\tableofcontents
\pagenumbering{arabic}

%%%%%%%%%%%%%%%%%%%%%%%%%%%%%%%%%%%%%%%%%%%%%%%%
%%%%%%%%%%%%%%%%%%%%%%%%%%%%%%%%%%%%%%%%%%%%%%%%
\paragraph*{Abstract}
%%%%%%%%%%%%%%%%%%%%%%%%%%%%%%%%%%%%%%%%%%%%%%%%
%%%%%%%%%%%%%%%%%%%%%%%%%%%%%%%%%%%%%%%%%%%%%%%%

This report reviews uncollided flux techniques (first and last collision methods) to be implemented in 
the Rattlesnake $S_N$ code in order to mitigate ray effects in modeling the TREAT reactor+hodoscope system. Angular
discretization techniques ($S_N$ and $P_N$) for the transport equation are notoriously poor at capturing effectively
streaming effects. 

\pagebreak

%%%%%%%%%%%%%%%%%%%%%%%%%%%%%%%%%%%%%%%%%%%%%%%%
%%%%%%%%%%%%%%%%%%%%%%%%%%%%%%%%%%%%%%%%%%%%%%%%
\section{Introduction}
%%%%%%%%%%%%%%%%%%%%%%%%%%%%%%%%%%%%%%%%%%%%%%%%
%%%%%%%%%%%%%%%%%%%%%%%%%%%%%%%%%%%%%%%%%%%%%%%%

\par
This report reviews uncollided flux techniques (first and last collision methods) to be implemented in 
the Rattlesnake $S_N$ code in order to mitigate ray effects in modeling the TREAT reactor+hodoscope system. Angular
discretization techniques ($S_N$ and $P_N$) for the transport equation are notoriously poor at capturing accurately
streaming effects. The uncollided component of the angular flux solution is the most anisotropic part and it can be
difficult for discrete-ordinate methods to accurately represent it; this phenomena is well-known and termed ``ray effects''.
However, it has long been recognized that an analytical or semi-analytical treatment of the uncollided flux, coupled with a discrete-ordinate treatment of the collided flux, can yield dramatic improvements in accuracy and computational efficiency. In this report, we present an algorithm for semi-analytical calculation of the uncollided flux. The algorithm seeks to compute
\begin{enumerate}
\item The existing uncollided flux at each point in a collect of point on the surface of the problem domain, with the angular flux reported for each ``source point''. The direction is fro the source point to the surface point. The source points are chosen to be spatial quadrature points that integrate the source volume.
\item Spatial and angular distributions of the uncollided flux in every cell in the problem. These provide all of the information necessary to form the first-collision source in each cell.
\end{enumerate}
The proposed algorithm is tailored for parallel efficiency given a spatial domain decomposition.

%%%%%%%%%%%%%%%%%%%%%%%%%%%%%%%%%%%%%%%%%%%%%%%%
\subsection{Basic Approach}
%%%%%%%%%%%%%%%%%%%%%%%%%%%%%%%%%%%%%%%%%%%%%%%%

For simplicity, we describe the uncollided flux technique using a single-speed description. Generalization to multigroup is straightforward.
We recall the one-group transport equation below:
\begin{equation}
\label{eq:transport_eq}
\vec{\Omega} \cdot \grad \Psi(\vec{r},\vec{\Omega}) + \sigma_t(\vec{r}) \Psi(\vec{r},\vec{\Omega}) = 
\sum_\ell \frac{2\ell+1}{4\pi}\sigma_{s,\ell}(\vec{r}) \sum_{m=-\ell}^{\ell} \Phi_{\ell,m}(\vec{r})Y_{\ell,m}(\vec{\Omega}) 
+ q(\vec{r},\vec{\Omega}) \,,
\end{equation}
where $\Psi(\vec{r},\vec{\Omega})$ is the angular flux at position $\vec{r}$ and in direction $\vec{\Omega}$, $Y_{\ell,m}(\vec{\Omega}) $ is the spherical harmonic function of degree $\ell$ and order $m$, $\Phi_{\ell,m}$ is the flux moment of degree $\ell$ and order $m$
\[
 \Phi_{\ell,m}(\vec{r}) = \int_{4\pi} d\Omega  \Psi(\vec{r},\vec{\Omega}) Y_{\ell,m}(\vec{\Omega}) \,.
\]
It helps to introduce an operator notation for brevity:
\begin{equation}
\label{eq:transport_eq_operator}
L\Psi = H\Psi +q \,.
\end{equation}
where $L$ is the streaming and total interaction operator and $H$ the scattering operator. 
Typically, the transport equation is solved iteratively (index $k$)
\begin{equation}
L\Psi^{(k+1)} = H\Psi^{(k)} +q \,.
\end{equation}

Let us now introduce a decomposition of the angular flux into collided and uncollided components:
\[
\Psi = \Psi^u + \Psi^c
\]
Then, \eqt{eq:transport_eq_operator} can be re-cast as
\begin{subequations}
\begin{equation}
\label{eq:_transport_eq_uncoll}
L^{RT}\Psi^u = q 
\end{equation}
\begin{equation}
\label{eq:_transport_eq_coll}
L^{S_N}\Psi^c = H\Psi^c + H\Psi^u 
\end{equation}
\end{subequations}
where we have emphasized how the transport operator $L$ will be solved in each case. The subscript $RT$ denotes ray-tracing
while  $S_N$ stands for discrete-ordinate techniques.

\subsubsection{Uncollided Flux Equation}
%---------------------------------------
The uncollided flux equation (\eqt{eq:_transport_eq_uncoll})
\[
L^{RT}\Psi^u = q \quad \text{or} \quad 
\vec{\Omega} \cdot \grad \Psi^u(\vec{r},\vec{\Omega}) + \sigma_t(\vec{r}) \Psi^u(\vec{r},\vec{\Omega}) = 
 q(\vec{r},\vec{\Omega}) \,,
\]
can be re-written, along a given direction $\vec{\Omega}$, as
\begin{equation}
\label{eq:transport_eq_attenuation}
\frac{d\Psi^u(\vec{r},\vec{\Omega})}{ds} + \sigma_t(\vec{r}) \Psi^u(\vec{r},\vec{\Omega}) = 
 q(\vec{r},\vec{\Omega}) \,,
\end{equation}
where $\vec{r} = \vec{r}_0+s\vec{\Omega}$ and $\vec{r}_0$ is an origin point (i.e., a source point). 
\eqt{eq:transport_eq_attenuation} can be solved analytically for simple geometries or semi-analytically 
using ray-tracing, for more complicated geometries.

\subsubsection{Collided Flux Equation}
%---------------------------------------

Once \eqt{eq:transport_eq_attenuation} has been solved, the uncollided flux solution $\Psi^u$ 
is thus available throughout the domain and the first collision source $q^{1st}$ can be computed as follows:
\[
q^{1st}(\vec{r},\vec{\Omega}) = H \Psi^u = \sum_\ell \frac{2\ell+1}{4\pi}\sigma_{s,\ell}(\vec{r}) \sum_{m=-\ell}^{\ell} \Phi^u_{\ell,m}(\vec{r})Y_{\ell,m}(\vec{\Omega}) 
\]
where $\Phi^u$ denote the flux moments computed from the uncollided angular flux $\Psi^u$. 
Then, one simply needs to solve \eqt{eq:_transport_eq_coll}, the equation for the collided component $\Psi^c$:
\begin{equation}
\label{eq:transport_eq_collided}
L^{S_N}\Psi^c = H\Psi^c + H\Psi^u = L^{S_N}\Psi^c = H\Psi^c + q^{1st} \,.
\end{equation}
Note that \eqt{eq:transport_eq_collided} is similar in nature to \eqt{eq:transport_eq}. An iterative technique (Source Iteration) is employed:
\[
L^{S_N}\Psi^{c,(k+1)} = H\Psi^{c,(k)} + q^{1st} \,.
\]
\eqt{eq:transport_eq_collided} gives the collided component of the angular flux. Standard discrete-ordinates methods are used to solve \eqt{eq:transport_eq_collided}.


\subsubsection{From First-collision Source Treatment to Last-collision Source Treatment}
%---------------------------------------


The above uncollided flux treatment addresses ray effect issues in the uncollided flux solution. However, ray effects can also occur in the collision flux solution, for instance, when there is a large distance from a small scatter source to a detector (e.g., neutrons streaming in a duct). TREAT's fuel motion monitoring systems clearly falls into this category. In a manner analogous to the uncollided flux treatment, a last collided flux treatment can be employed to remove ray effects in a detector response. The last collision source treatment is simply expressed as the original transport equation, where the first-collided source due to the uncollided angular flux and the collided source (obtained from a discrete ordinate solution) 
are known (they come from the ray-tracing and the discrete-ordinate solution, previously described):
\begin{equation}
\label{eq_transport_last}
L^{RT}\Psi^{last} = H\Psi^c + H\Psi^u + q = q^{last}\,,
\end{equation}
That is, instead of using $\Psi^c + \Psi^u$ as the final angular flux at the detector locations, we perform one
more ray-tracing using the total source $q^{last}$ and then employ $\Psi^{last}$ as the final answer.
%given a first-collided and a multiple collided source (and the external source), one ray-traces the angular flux solution, from any point in the domain to the detector surface.
Note, we can also solve for the last collision angular flux $\Psi^{c,last}$ as follows
\begin{equation}
\label{eq_transport_last_coll}
L^{RT}\Psi^{c,last} = H\Psi^c + H\Psi^u  = q^{c,last}\,.
\end{equation}
\eqt{eq_transport_last_coll} is simply obtained by subtracting \eqt{eq_transport_last} and \eqt{eq:_transport_eq_uncoll}. Then
\[
\Psi^{last} = \Psi^{c,last}  + \Psi^u \,.
\]

%%%%%%%%%%%%%%%%%%%%%%%%%%%%%%%%%%%%%%%%%%%%%%%%
\subsection{Parallel Aspects of the Traditional Uncollided Flux Treatment}
%%%%%%%%%%%%%%%%%%%%%%%%%%%%%%%%%%%%%%%%%%%%%%%%

Most uncollided-flux algorithms track directly form each source point to a set of a few points in each cell 
of the domain. The source volume assigned to each source point is effectively treated as vanishingly small, so that 
the angular flux at any other spatial point is treated as a $\delta$-function in the direction $\vec{\Omega}$ subtended 
by the two points. 
This traditional algorithm does not scale well given a spatial domain decomposition: the process (or processor) that owns
a source point must execute $\mathcal{O}(N)$ work given $N$ spatial cells. If the process that owns the  source point
has knowledge of the entire domain, its works grows proportionally to $N$. When sources are localized (as this is the case very the experimental vehicle in TREAT or the detectors in the hodoscope n the case of the last-collision source treatment), 
then many processes not owning any source points stay idle. A simple workaround would be for processes not owning source points to ray-trace once a ray enters their subdomains (this would be mandatory is the mesh is truly parallel and not reproduced on any one processor). However, there are two deficiencies with this approach:
\begin{enumerate}
\item The algorithm is partially sequential. Processes not owning a source point stay idle until a ray has reached them.
\item As one moves further away from the source point, the ray density decreases, which can cause inaccuracies.
\end{enumerate}
\tcr{if time, I add a figure to illustrate this}


%%%%%%%%%%%%%%%%%%%%%%%%%%%%%%%%%%%%%%%%%%%%%%%%
%%%%%%%%%%%%%%%%%%%%%%%%%%%%%%%%%%%%%%%%%%%%%%%%
\section{SQUID: A MOOSE-based Ray-Tracing Application}
%%%%%%%%%%%%%%%%%%%%%%%%%%%%%%%%%%%%%%%%%%%%%%%%
%%%%%%%%%%%%%%%%%%%%%%%%%%%%%%%%%%%%%%%%%%%%%%%%

%%%%%%%%%%%%%%%%%%%%%%%%%%%%%%%%%%%%%%%%%%%%%%%%
\subsection{Overview}
%%%%%%%%%%%%%%%%%%%%%%%%%%%%%%%%%%%%%%%%%%%%%%%%
\tcr{a brief overview of Squid. What it is used for today (MOC, others?). many some scalability results.}

%%%%%%%%%%%%%%%%%%%%%%%%%%%%%%%%%%%%%%%%%%%%%%%%
\subsection{Enhancements Required for the Uncollided Flux Treatment}
%%%%%%%%%%%%%%%%%%%%%%%%%%%%%%%%%%%%%%%%%%%%%%%%
\tcr{check}
Current, SQUID ray-traces through a domain (mesh) and computes the angular flux attenuation using a given direction $\vec{\Omega}$, a starting point $\vec{r}_0$, and a termination distance $D:
\[
\Psi^u(\vec{r}_0+s\vec{\Omega},\vec{\Omega}) = 
\Psi^u(\vec{r}_0,\vec{\Omega}) 
\]

%%%%%%%%%%%%%%%%%%%%%%%%%%%%%%%%%%%%%%%%%%%%%%%%
%%%%%%%%%%%%%%%%%%%%%%%%%%%%%%%%%%%%%%%%%%%%%%%%
\section{Finite-element Evaluation of the First-collided Scattering Source}
%%%%%%%%%%%%%%%%%%%%%%%%%%%%%%%%%%%%%%%%%%%%%%%%
%%%%%%%%%%%%%%%%%%%%%%%%%%%%%%%%%%%%%%%%%%%%%%%%


%%%%%%%%%%%%%%%%%%%%%%%%%%%%%%%%%%%%%%%%%%%%%%%%
%%%%%%%%%%%%%%%%%%%%%%%%%%%%%%%%%%%%%%%%%%%%%%%%

%\clearpage
%\pagebreak
%\newpage
%%%%%%%%%%%%%%%%%%%%%%%%%%%%%%%%%%%%%%%%%%%%%%%%%%%%%%%%%%%%%%%%%%%%%%%%%%%%%
%%%%%%%%%%%%%%%%%%%%%%%%%%%%%%%%%%%%%%%%%%%%%%%%%%%%%%%%%%%%%%%%%%%%%%%%%%%%%
%\bibliographystyle{abbrvnat}
%\bibliography{references_IQS}
%%%%%%%%%%%%%%%%%%%%%%%%%%%%%%%%%%%%%%%%%%%%%%%%%%%%%%%%%%%%%%%%%%%%%%%%%%%%%
%%%%%%%%%%%%%%%%%%%%%%%%%%%%%%%%%%%%%%%%%%%%%%%%%%%%%%%%%%%%%%%%%%%%%%%%%%%%%
%\appendix
%%%%%%%%%%%%%%%%%%%%%%%%%%%%%%%%%%%%%%%%%%%%%%%%%%%%%%%%%%%%%%%%%%%%%%%%%%%%%
%%%%%%%%%%%%%%%%%%%%%%%%%%%%%%%%%%%%%%%%%%%%%%%%%%%%%%%%%%%%%%%%%%%%%%%%%%%%%

%%%%%%%%%%%%%%%%%%%%%%%%%%%%%%%%%%%%%%%%%%%%%%%%%%%%%%%%%%%%%%%%%%%%%%%%%%%%%%%%

\end{document}