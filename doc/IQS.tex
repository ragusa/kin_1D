%% Based on a TeXnicCenter-Template, which was
%% created by Christoph B�rensen
%% and slightly modified by Tino Weinkauf.
%%%%%%%%%%%%%%%%%%%%%%%%%%%%%%%%%%%%%%%%%%%%%%%%%%%%%%%%%%%%%

\documentclass[12pt]{scrartcl} %This is a special class provided by the KOMA script, which does a lot of adjustments to adapt the standard LaTeX classes to european habits, change to [a4paper,12pt,twoside] for doublesided layout


%########################### Preferences #################################


% ******** vmargin settings *********
\usepackage{vmargin} %This give you full control over the used page area, it maybe not the idea od Latex to do so, but I wanted to reduce to amount of white space on the page
%\setpapersize{A4}
%\setmargins{3.5cm}%			%linker Rand, left edge
					 %{1.5cm}%     %oberer Rand, top edge
           %{14.7cm}%		%Textbreite, text width
           %{23.42cm}%   %Texthoehe, text hight
           %{14pt}%			%Kopfzeilenh�he, header hight
           %{1cm}%   	  %Kopfzeilenabstand, header distance
           %{0pt}%				%Fu�zeilenhoehe footer hight
           %{2cm}%    	  %Fusszeilenabstand, footer distance         

% ********* Font definiton ************
%\usepackage{t1enc} % as usual
%\usepackage[latin1]{inputenc} % as usual
%\usepackage{times}		
%\usepackage{mathptmx}  	%mathematical fonts for use with times, I encountered some problems using this package togather with pdftex, which I was not able to resolve

% ********* Graphics definition *******
%\usepackage[pdftex]{graphicx} % required to import graphic files
\usepackage{color} %allows to mark some entries in the tables with color
%\usepackage{eso-pic} % these two are required to add the little picture on top of every page
%\usepackage{everyshi} % these two are required to add the little picture on top of every page
%\renewcommand{\floatpagefraction}{0.7} %default:0.5 allows two big pictures on one page

\usepackage{amsthm}
\usepackage{amssymb}
\usepackage{amsmath}
\usepackage{amsfonts}
\usepackage{amstext}
\usepackage{amsbsy}
\usepackage{mathbbol} 
%\usepackage{pmatrix} 

%********** Enybeling Hyperlinks *******
%\usepackage[pdfborder=000,pdftex=true]{hyperref}% this enables jumping from a reference and table of content in the pdf file to its target

% ********* Table layout **************
\usepackage{booktabs}	  	%design of table, has an excellent documentation
%\usepackage{lscape}			%use this if you want to rotate the table together with the lines around the table

%  new definitions
\renewcommand{\div}{\bs{\nabla}\! \cdot \!}
\newcommand{\grad}{\bs{\nabla}}
% extra space
\newcommand{\qq}{\quad\quad}
% common reference commands
\newcommand{\eqt}[1]{Eq.~(\ref{#1})}                     % equation
\newcommand{\fig}[1]{Fig.~\ref{#1}}                      % figure
\newcommand{\tbl}[1]{Table~\ref{#1}}                     % table
\newcommand{\sct}[1]{Section~\ref{#1}}                   % section
\newcommand{\app}[1]{Appendix~\ref{#1}}                   % appendix

\newcommand{\bs}[1]{\mathbf{#1}}
\newcommand{\dd}{\mathrm{d}}

\newcommand{\be}{\begin{equation}}
\newcommand{\ee}{\end{equation}}
\newcommand{\vn}{\vec{n}}
\newcommand{\vel}{\vec{\mathrm{v}}}
\newcommand{\adj}{\Phi^\dagger_0}

% ********* Caption Layout ************
\usepackage{ccaption} % allows special formating of the captions
\captionnamefont{\bf\footnotesize\sffamily} % defines the font of the caption name (e.g. Figure: or Table:)
\captiontitlefont{\footnotesize\sffamily} % defines the font of the caption text (same as above, but not bold)
\setlength{\abovecaptionskip}{0mm} %lowers the distace of captions to the figure


% ********* Header and Footer **********
% This is something to play with forever. I use here the advanced settings of the KOMA script

\usepackage{scrpage2} %header and footer using the options for the KOMA script
\renewcommand{\headfont}{\footnotesize\sffamily} % font for the header
\renewcommand{\pnumfont}{\footnotesize\sffamily} % font for the pagenumbers

%the following lines define the pagestyle for the main document
\defpagestyle{cb}{%
(\textwidth,0pt)% sets the border line above the header
{\pagemark\hfill\headmark\hfill}% doublesided, left page
{\hfill\headmark\hfill\pagemark}% doublesided, right page
{\hfill\headmark\hfill\pagemark}%  onesided
(\textwidth,1pt)}% sets the border line below the header
%
{(\textwidth,1pt)% sets the border line above the footer
{{\it Jean Ragusa}\hfill TAMU}% doublesided, left page
{Jean Ragusa\hfill{\it TAMU}}% doublesided, right page
{Jean Ragusa\hfill{\it TAMU}} % one sided printing
(\textwidth,0pt)% sets the border line below the footer
}

%this defines the page style for the first pages: all empty
\renewpagestyle{plain}%
	{(\textwidth,0pt)%
		{\hfill}{\hfill}{\hfill}%
	(\textwidth,0pt)}%
	{(\textwidth,0pt)%	
		{\hfill}{\hfill}{\hfill}%
	(\textwidth,0pt)}

%********** Footnotes **********
\renewcommand{\footnoterule}{\rule{5cm}{0.2mm} \vspace{0.3cm}} %increases the distance of footnotes from the text
\deffootnote[1em]{1em}{1em}{\textsuperscript{\normalfont\thefootnotemark}} %some moe formattion on footnotes

\renewcommand{\labelitemii}{$\diamond$}

%%%%%%%%%%%%%%%%%%%%%%%%%%%%%%%% boxes
\usepackage{color}
\definecolor{myblue}{rgb}{.8, .8, 1}
\usepackage{empheq}

\newlength\mytemplen
\newsavebox\mytempbox

\makeatletter
\newcommand\mybluebox{%
    \@ifnextchar[%]
       {\@mybluebox}%
       {\@mybluebox[0pt]}}

\def\@mybluebox[#1]{%
    \@ifnextchar[%]
       {\@@mybluebox[#1]}%
       {\@@mybluebox[#1][0pt]}}

\def\@@mybluebox[#1][#2]#3{
    \sbox\mytempbox{#3}%
    \mytemplen\ht\mytempbox
    \advance\mytemplen #1\relax
    \ht\mytempbox\mytemplen
    \mytemplen\dp\mytempbox
    \advance\mytemplen #2\relax
    \dp\mytempbox\mytemplen
    \colorbox{myblue}{\hspace{1em}\usebox{\mytempbox}\hspace{1em}}}

\makeatother

%################ End Preferences, Begin Document #####################

\pagestyle{plain} % on headers or footers on the first page

%%%%%%%%%%%%%%%%%%%%%%%%%%%%%%%%%%%%%%%%%%%%%%%%%%%%%%%%%%%%%%%%%%%%%%%%%%%%%
%%%%%%%%%%%%%%%%%%%%%%%%%%%%%%%%%%%%%%%%%%%%%%%%%%%%%%%%%%%%%%%%%%%%%%%%%%%%%
\begin{document}
%%%%%%%%%%%%%%%%%%%%%%%%%%%%%%%%%%%%%%%%%%%%%%%%%%%%%%%%%%%%%%%%%%%%%%%%%%%%%
%%%%%%%%%%%%%%%%%%%%%%%%%%%%%%%%%%%%%%%%%%%%%%%%%%%%%%%%%%%%%%%%%%%%%%%%%%%%%

\begin{center}

\begin{figure}[th]
    \centering
		%\includegraphics[width=10cm]{logo.jpg}
	\label{fig:logo}
\end{figure}

\vspace{2cm}
\vspace{2cm}

% There might be better solutions for the title page, giving all distances and sizes manually was simply the easiest solution

{\Huge\bf\sf IQS Notes}

\vspace{.5cm}

{\Huge\bf\sf TAMU}

\vspace{.5cm}

%{\Huge\bf\sf to use with TeXnicCenter}

\vspace{2cm}

{\Large\bf\sf Jean C. Ragusa}
\vspace{2cm}

{\Large\bf\sf \today} %adds the current date

\vspace{\fill}

\tt{jean.ragusa@tamu.edu}

\end{center}
\newpage

%%The following loads the picture on top of every page, the numbers in \put() define the position on the page:
%\AddToShipoutPicture{\setlength\unitlength{0.1mm}\put(604,2522){\includegraphics[width=1.5cm]{logo.jpg}}}

\pagestyle{cb} % now we want to have headers and footers

\tableofcontents

\newpage


%%%%%%%%%%%%%%%%%%%%%%%%%%%%%%%%%%%%%%%%%%%%%%%%%%%%%%%%%%%%%%%%%%%%%%%%%%%%%
%%%%%%%%%%%%%%%%%%%%%%%%%%%%%%%%%%%%%%%%%%%%%%%%%%%%%%%%%%%%%%%%%%%%%%%%%%%%%
\section{Time-dependent diffusion}
%%%%%%%%%%%%%%%%%%%%%%%%%%%%%%%%%%%%%%%%%%%%%%%%%%%%%%%%%%%%%%%%%%%%%%%%%%%%%
%%%%%%%%%%%%%%%%%%%%%%%%%%%%%%%%%%%%%%%%%%%%%%%%%%%%%%%%%%%%%%%%%%%%%%%%%%%%%

\be
\frac{1}{v} \frac{\partial \Phi }{\partial t} = \nu_p \sigma_f \Phi - \left( -\div D \grad \Phi + \sigma_a \Phi \right) + \lambda C
\ee
\be
\frac{dC}{dt} = \nu_d \sigma_f \Phi - \lambda C
\ee

%%%%%%%%%%%%%%%%%%%%%%%%%%%%%%%%%%%%%%%%%%%%%%%%%%%%%%%%%%%%%%%%%%%%%%%%%%%%%
%%%%%%%%%%%%%%%%%%%%%%%%%%%%%%%%%%%%%%%%%%%%%%%%%%%%%%%%%%%%%%%%%%%%%%%%%%%%%
\section{IQS}
%%%%%%%%%%%%%%%%%%%%%%%%%%%%%%%%%%%%%%%%%%%%%%%%%%%%%%%%%%%%%%%%%%%%%%%%%%%%%
%%%%%%%%%%%%%%%%%%%%%%%%%%%%%%%%%%%%%%%%%%%%%%%%%%%%%%%%%%%%%%%%%%%%%%%%%%%%%


%%%%%%%%%%%%%%%%%%%%%%%%%%%%%%%%%%%%%%%%%%%%%%%%%%%%%%%%%%%%%%%%%%%%%%%%%%%%%
\subsection{Factorization}
%%%%%%%%%%%%%%%%%%%%%%%%%%%%%%%%%%%%%%%%%%%%%%%%%%%%%%%%%%%%%%%%%%%%%%%%%%%%%
Factorization into amplitude and shape functions:
\be
\Phi(\vec{r},t) = p(t) \varphi(\vec{r},t)
\ee

Factorization is not unique:
\[
\Phi(\vec{r},t) = \frac{p(t)}{f(t)} \left( f(t) \varphi(\vec{r},t) \right)
\]

%%%%%%%%%%%%%%%%%%%%%%%%%%%%%%%%%%%%%%%%%%%%%%%%%%%%%%%%%%%%%%%%%%%%%%%%%%%%%
\subsection{Equations for the shape $\varphi$}
%%%%%%%%%%%%%%%%%%%%%%%%%%%%%%%%%%%%%%%%%%%%%%%%%%%%%%%%%%%%%%%%%%%%%%%%%%%%%

\be
\frac{1}{v} \frac{\partial \varphi }{\partial t} = \nu_p \sigma_f \varphi - \left( -\div D \grad \varphi + (\sigma_a+\alpha(t)) \varphi \right) + \frac{1}{p}\lambda C
\ee
where
\[
\alpha(t) = \frac{1}{v} \frac{1}{p(t)}\frac{dp}{dt}
\]
Very little change in the precursor equation:
\be
\frac{dC}{dt} = \nu_d \sigma_f p \varphi - \lambda C
\ee


%%%%%%%%%%%%%%%%%%%%%%%%%%%%%%%%%%%%%%%%%%%%%%%%%%%%%%%%%%%%%%%%%%%%%%%%%%%%%
\subsection{Equations for the amplitude $p$}
%%%%%%%%%%%%%%%%%%%%%%%%%%%%%%%%%%%%%%%%%%%%%%%%%%%%%%%%%%%%%%%%%%%%%%%%%%%%%
These are simply obtained by integration of the shape equations over the entire space. We use the following short-cut notation:
\[
\langle f, g \rangle = \int_{\text{domain}} f(\vec{r})g(\vec{r}) d^3r
\]
The starting point is the shape equation (re-written below before division by $p$ occurred).
\be
\frac{1}{v} p \frac{\partial \varphi }{\partial t} + \frac{1}{v} \varphi \frac{dp}{dt} = \nu_p \sigma_f p \varphi - p \left( -\div D \grad \varphi + \sigma_a \varphi \right) + \lambda C
\ee
Before carrying out the integration over space, we multiply the equation by a time-independent weighting function (a common choice for this function is the initial adjoint flux, $\adj$).

After a bit of algebra, we obtain
\be
\frac{dp}{dt} = \left(\frac{\rho}{\Lambda}-\frac{\beta}{\Lambda} - \theta \right) p + \lambda \xi
\ee
\be
\frac{d\xi}{dt} = \frac{\beta}{\Lambda} p - \left( \lambda + \theta \right) \xi
\ee 
with the following definitions
\begin{align}
\frac{\rho}{\Lambda} &= \frac{\langle \adj,  \left[ \nu \sigma_f - \left( -\div D \grad  + \sigma_a  \right)\right] \varphi \rangle}{\langle \adj, v^{-1} \,\varphi \rangle}\\
\frac{\beta}{\Lambda}&= \frac{\langle \adj,  \nu_d \sigma_f \varphi \rangle}{\langle \adj, v^{-1} \,\varphi \rangle}\\
\theta               &= \frac{\langle \adj,  v^{-1} \frac{\partial \varphi}{\partial t} \rangle}{\langle \adj, v^{-1} \,\varphi \rangle}
                      = \frac{\frac{d}{dt}\langle \adj,  v^{-1} \varphi \rangle}{\langle \adj, v^{-1} \,\varphi \rangle}
\end{align}
The total time derivative expression for $\theta$ is due to the fact that $v$ and the initial adjoint function are both independent of time.  

We have also defined
\[
\xi = \frac{ \langle \adj, C \rangle}{\langle \adj, v^{-1} \,\varphi \rangle}
\]
which gives
\[
\langle \adj, \frac{dC}{dt} \rangle = \langle \adj, v^{-1} \,\varphi \rangle \frac{d \xi}{dt} + \xi \frac{d}{dt}\langle \adj, v^{-1} \,\varphi \rangle
\]
and explains the presence of the $\theta$ term in the PRKE equation for the precursor concentrations.

%%%%%%%%%%%%%%%%%%%%%%%%%%%%%%%%%%%%%%%%%%%%%%%%%%%%%%%%%%%%%%%%%%%%%%%%%%%%%
\subsection{Normalization choice}
%%%%%%%%%%%%%%%%%%%%%%%%%%%%%%%%%%%%%%%%%%%%%%%%%%%%%%%%%%%%%%%%%%%%%%%%%%%%%

We choose to have the following quantity constant in time
\be
\langle \adj, v^{-1} \,\varphi \rangle = K_0
\ee
even though the shape itself may still be allowed to change in time. 

This makes the factorization {\bf unique} and lets $\theta=0$, yielding the standard form of the PRKE equations (note that
in the PRKE approximation, i.e., when the shape is assume independent of time, we automatically have $\theta=0$).

\be
\frac{dp}{dt} = \left(\frac{\rho}{\Lambda}-\frac{\beta}{\Lambda} \right) p + \lambda \xi
\ee
\be
\frac{d\xi}{dt} = \frac{\beta}{\Lambda} p - \lambda \xi
\ee 

%%%%%%%%%%%%%%%%%%%%%%%%%%%%%%%%%%%%%%%%%%%%%%%%%%%%%%%%%%%%%%%%%%%%%%%%%%%%%
\subsection{IQS solution procedure}
%%%%%%%%%%%%%%%%%%%%%%%%%%%%%%%%%%%%%%%%%%%%%%%%%%%%%%%%%%%%%%%%%%%%%%%%%%%%%

In the IQS method, the shape is solved for on macro time steps $[t^n,t^{n+1}]$ (one solve for the shape over that time step) while the 
PRKE equations are solve on micro time steps: $t^n, t^{n+1/M}, \ldots, t^{n+m/M}, \ldots, t^{n+1}$ (for a total of $M$ solves 
during one macro time step).

Because of the factorization approximation, we are actually solving a nonlinear problem over the macro time step. Convergence is said to be reached 
when the new shape $\varphi(\vec{r},t^{n+1})$ satisfies the normalize condition $\langle \adj, v^{-1} \,\varphi \rangle = K_0$.

The process over $[t^n,t^{n+1}]$ is as follows. Let the nonlinear iteration index be $\ell$.
\begin{enumerate}
\item Set $\ell=0$
\item Estimate the shape at the end time $\varphi^{n+1,\ell}$ (could be extrapolation, or just assume equal to begin time shape function)
\item Using the begin time and end time shape to compute $\rho/\Lambda$ and $\beta/\Lambda$ at the micro time steps 
(this is a bit costly, so in reality we only do this for a certain multiple of the micro time steps, assuming a linear variation in time of the shape and using the exact values of the XS at the evaluation instants). 
\item Solve the PRKE equations using data from 3.
\item Compute $\alpha(t^{n+1}) = \frac{1}{v} \frac{1}{p(t^{n+1})}\left.\frac{dp}{dt}\right|_{t^{n+1}}$
\item Compute the shape at the end time $\varphi^{n+1,\ell+1}$
\item Check for convergence in $K_0$. If so, EXIT
\item Otherwise $\ell \leftarrow \ell+1$ and GO BACK to 3.
\end{enumerate}

%%%%%%%%%%%%%%%%%%%%%%%%%%%%%%%%%%%%%%%%%%%%%%%%%%%%%%%%%%%%%%%%%%%%%%%%%%%%%
%%%%%%%%%%%%%%%%%%%%%%%%%%%%%%%%%%%%%%%%%%%%%%%%%%%%%%%%%%%%%%%%%%%%%%%%%%%%%
\section{Homogeneous perturbation in an homogenous domain}
%%%%%%%%%%%%%%%%%%%%%%%%%%%%%%%%%%%%%%%%%%%%%%%%%%%%%%%%%%%%%%%%%%%%%%%%%%%%%
%%%%%%%%%%%%%%%%%%%%%%%%%%%%%%%%%%%%%%%%%%%%%%%%%%%%%%%%%%%%%%%%%%%%%%%%%%%%%

In this Section I consider a ``simple'' test case which I was hoping to test the 
implementation of IQS.

%%%%%%%%%%%%%%%%%%%%%%%%%%%%%%%%%%%%%%%%%%%%%%%%%%%%%%%%%%%%%%%%%%%%%%%%%%%%%
\subsection{Problem description}
%%%%%%%%%%%%%%%%%%%%%%%%%%%%%%%%%%%%%%%%%%%%%%%%%%%%%%%%%%%%%%%%%%%%%%%%%%%%%

Consider an initially critical homogeneous slab reactor solved with 1-g diffusion theory.
The initial flux/shape is therefore a cosine. We can make the reactor critical simply by dividing
the $\nu$'s with k$_\text{eff}$.

At time $t=0$, the $\sigma_a$ is subjected to a variation that is linear in time (and this, at all times,
the domain remains homogenous, i.e., no spatially varying cross sections).

Because the initial flux only contains the steady-state fundamental mode and because the modification is homogeneous,
no other modes are excited and the flux function only changes in magnitude. That is, the shape function remains constant
and the flux amplitude is actually given by the solution of the PRKE.

%%%%%%%%%%%%%%%%%%%%%%%%%%%%%%%%%%%%%%%%%%%%%%%%%%%%%%%%%%%%%%%%%%%%%%%%%%%%%
\subsection{Brute force temporal discretization}
%%%%%%%%%%%%%%%%%%%%%%%%%%%%%%%%%%%%%%%%%%%%%%%%%%%%%%%%%%%%%%%%%%%%%%%%%%%%%

A backward Euler time discretization yields, for the flux,

\be
\label{eq:ex1}
\frac{1}{v} \frac{\Phi^{n+1}-\Phi^n }{\Delta t} = (\nu_p \sigma_f)^{n+1} \Phi^{n+1} - \left( -\div D \grad + \sigma_a \right)^{n+1} \Phi^{n+1} + \lambda C^{n+1}
\ee
For a constant shape, this becomes:
\be
\frac{\varphi_0}{v} \frac{p^{n+1}-p^n }{\Delta t} = (\nu_p \sigma_f)^{n+1} \varphi_0 p^{n+1} - \left( -\div D \grad + \sigma_a \right)^{n+1} \varphi_0 p^{n+1} + \lambda \varphi_0 \xi^{n+1}
\ee
What equations should be satisfied by $p^{n+1}$ and $\xi^{n+1}$? Simply the PRKE discretized with backward Euler over $[t^n,t^{n+1}]$ (one step!). This
can be easily verified if you take the above equation, multiply it by $\adj$ and integrate over the entire domain.

We numerically observed that the brute-force solve yielded the same results as the PRKE solve when using the initial shape in the PRKE parameters.

%%%%%%%%%%%%%%%%%%%%%%%%%%%%%%%%%%%%%%%%%%%%%%%%%%%%%%%%%%%%%%%%%%%%%%%%%%%%%
\subsection{IQS approach with NO subcycling for the PRKE}
%%%%%%%%%%%%%%%%%%%%%%%%%%%%%%%%%%%%%%%%%%%%%%%%%%%%%%%%%%%%%%%%%%%%%%%%%%%%%
If the shape and the amplitude functions are solved using the same time step, then we have

\be
\label{eq:ex2}
\frac{1}{v} \frac{\varphi^{n+1} -\varphi^n}{\Delta t} = (\nu_p \sigma_f)^{n+1} \varphi^{n+1} - \left( -\div D \grad + (\sigma_a+\alpha) \right)^{n+1} \varphi^{n+1} + \frac{1}{p^{n+1}}\lambda C^{n+1}
\ee
where 
\[
\alpha^{n+1} = \frac{1}{v} \frac{1}{p^{n+1}}\frac{p^{n+1}-p^{n}}{\Delta t}
\]
This form of $\alpha$ is due to the fact that we assumed here to employ only one time interval for the PRKE (no subcycling).
Note that by letting $\varphi^{n+1}=\varphi^n=\varphi_0$,  we can re-arrange \eqt{eq:ex2} as \eqt{eq:ex1}.

%%%%%%%%%%%%%%%%%%%%%%%%%%%%%%%%%%%%%%%%%%%%%%%%%%%%%%%%%%%%%%%%%%%%%%%%%%%%%
\subsection{IQS approach with subcycling for the PRKE}
%%%%%%%%%%%%%%%%%%%%%%%%%%%%%%%%%%%%%%%%%%%%%%%%%%%%%%%%%%%%%%%%%%%%%%%%%%%%%

Now, \eqt{eq:ex2} tells you also something quite important. If you solve \eqt{eq:ex2} for $\varphi^{n+1}$, you obtain a solution (we verified if 
the PRKE are done with the same time step as the shape, that the shape solution remains constant). Now, 
\eqt{eq:ex2} is a linear system ($A \varphi^{n+1} = b$), so if I change slightly the entry of matrix $A$, I should get another solution for 
$\varphi^{n+1}$. Well, subcycling the PRKE gives us a better value for $\alpha$ (it is more accurate since we solve the PRKE on a finer time grid),
so we automatically see that $\varphi^{n+1}$ cannot remain constant equal to its initial value. Where's the flaw ????


%%%%%%%%%%%%%%%%%%%%%%%%%%%%%%%%%%%%%%%%%%%%%%%%%%%%%%%%%%%%%%%%%%%%%%%%%%%%%
\end{document}
%%%%%%%%%%%%%%%%%%%%%%%%%%%%%%%%%%%%%%%%%%%%%%%%%%%%%%%%%%%%%%%%%%%%%%%%%%%%%
%%%%%%%%%%%%%%%%%%%%%%%%%%%%%%%%%%%%%%%%%%%%%%%%%%%%%%%%%%%%%%%%%%%%%%%%%%%%%



