\documentstyle[11pt]{letter}

%%%%%% Letter Size Setup %%%%%%%%%%%%%%%%%%%%%%%%%%%%%%%%%%%%%%%%%%%%%%%%%%
%        \addtolength{\textwidth}{2.5cm}     %% For longer or shorter text width
        \addtolength{\topmargin}{-3.5cm}    %% For more or less top margin
       \addtolength{\textheight}{7cm}    %% For longer or shorter textheight
%        \addtolength{\oddsidemargin}{-1.25cm} %% For odd side margin (twoside)
                                            %% or margin (oneside)

\address{Zachary M. Prince\\ 
Department of Nuclear Engineering \\
Texas A\&M University\\
College Station, TX 77843-3133, USA\\
phone: (360) 362 6622\\
e-mail: zachmprince@tamu.edu \vspace{0.5cm}}

%%%%%% The Signature  and Date %%%%%%%%%%%%%%%%%%%%%%%%%%%%%%%%%%%%%%%%%%%%

\signature{\vspace{-1.25cm}Zachary M. Prince, Jean C. Ragusa}   


\begin{document}

\begin{letter}{Professor Sara A. Pozzi\\
    Editor,\\
    Elsevier Annals of Nuclear Energy}
\date{\today}
%%%%%% More vertical space can be added here %%%%%%%%%%%%%%%%%%%%%%%%%%%%%%
%         \vspace{3.0cm}

\opening{Dear Professor Pozzi,}
         \vspace{0.25cm}
%%%%%% More vertical space can be added here %%%%%%%%%%%%%%%%%%%%%%%%%%%%%%

Please find attached a copy of our manuscript titled ``Multiphysics Reactor-core Simulations Using the Improved Quasi-Static Method'' for submission to the {\em Annals of Nuclear Energy}. 

In this paper, we revisit the improved quasi-static method (IQS) for its application to high-order time discretizations and temperature feedback dynamics. We derive the IQS and IQS predictor-corrector method (IQS-PC) with the semi-analytical treatment of delayed neutron precursors and adiabatic heat-up. Temperature computation and feedback are treated on a separate time scale from the neutronic shape and amplitude functions to increase the efficiency of nonlinear quasi-static process.

We show numerical results from four different test cases where we compare error norms from IQS computation and standard implicit discretization of the flux equations. A one-dimensional test was constructed to analyze error and nonlinear iteration convergence for IQS with up to fourth-order backward-difference-formulae discretization. IQS was applied to the TWIGL benchmark to show its performance with step doubling time adaptation. Analysis of the quasi-static treatment of temperature feedback are shown with results from the LRA benchmark and a test case from a model of the Transient Reactor Testing Facility (TREAT). 

IQS and IQS-PC were implemented into the MOOSE-Rattlesnake framework of Idaho National Laboratory.

Thank you for considering this manuscript for publication in the {\it Annals of Nuclear Energy}.

\vspace{0.25cm}


%%%%%% More vertical space can be added here %%%%%%%%%%%%%%%%%%%%%%%%%%%%%%

%%%%%%% The Closing %%%%%%%%%%%%%%%%%%%%%%%%%%%%%%%%%%%%%%%%%%%%%%%%%%%%%%%
\closing{Best regards, }

\end{letter}
\end{document}

