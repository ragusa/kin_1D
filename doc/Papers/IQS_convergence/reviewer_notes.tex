\documentclass[11pt]{letter}

%%%%%% Letter Size Setup %%%%%%%%%%%%%%%%%%%%%%%%%%%%%%%%%%%%%%%%%%%%%%%%%%
        \setlength{\textwidth}{6in}     %% For longer or shorter text width
        \setlength{\topmargin}{0in}    %% For more or less top margin
       \setlength{\textheight}{8.6in}    %% For longer or shorter textheight
        \setlength{\oddsidemargin}{0.25in} %% For odd side margin (twoside)
                                            %% or margin (oneside)
        \setlength{\paperwidth}{8.5in}
        \setlength{\paperheight}{11in}
        \setlength{\topmargin}{0pt}
        \setlength{\headheight}{0pt}
        \setlength{\headsep}{0pt}
        \setlength{\footskip}{0.4in}
        \setlength{\marginparwidth}{0in}
        \setlength{\marginparsep}{0in}
        \setlength{\hoffset}{0in}
        \setlength{\voffset}{0in}

\address{Zachary M. Prince\\ 
Department of Nuclear Engineering \\
Texas A\&M University\\
College Station, TX 77843-3133, USA\\
phone: (360) 362 6622\\
e-mail: zachmprince@tamu.edu \vspace{0.5cm}}

%%%%%% The Signature  and Date %%%%%%%%%%%%%%%%%%%%%%%%%%%%%%%%%%%%%%%%%%%%

\signature{\vspace{-1.25cm}Zachary M. Prince, Jean C. Ragusa}   

\usepackage{xspace} 

\newcommand{\iqspc}{IQS-PC\xspace}
\newcommand{\section}[1]{\textbf{#1}}

%\newcommand{\fix}{\textsquare}
%\newcommand{\working}{$\boxdot$}
%\newcommand{\nofix}{?}
%\newcommand{\done}{\checkmark}
\newcommand{\fix}{$\bullet$}
\newcommand{\working}{$\bullet$}
\newcommand{\nofix}{$\bullet$}
\newcommand{\done}{$\bullet$}

%\newcommand{\easy}[1]{\textcolor{ForestGreen}{#1}}
%\newcommand{\medm}[1]{\textcolor{BurntOrange}{#1}}
%\newcommand{\hard}[1]{\textcolor{Red}{#1}}
\newcommand{\easy}[1]{{\textit{#1}}}
\newcommand{\medm}[1]{{\textit{#1}}}
\newcommand{\hard}[1]{{\textit{#1}}}

%\usepackage{color}
%\newcommand{\tcr}[1]{\textcolor{red}{#1}}

%%%%%%%%%%%%%%%%%%%%%%%%
\begin{document}
%%%%%%%%%%%%%%%%%%%%%%%%

\begin{letter}{Professor Sara A. Pozzi\\
    Editor,\\
    Elsevier Annals of Nuclear Energy\\
    \bigskip
    Re: Resubmission of manuscript ``Multiphysics Reactor-core Simulations Using the Improved Quasi-Static Method'', ANUCENE-D-18-00462}
\date{\today}
%%%%%% More vertical space can be added here %%%%%%%%%%%%%%%%%%%%%%%%%%%%%%
%         \vspace{3.0cm}

\opening{Dear Professor Pozzi,}
         \vspace{0.25cm}
%%%%%% More vertical space can be added here %%%%%%%%%%%%%%%%%%%%%%%%%%%%%%

Thank you for the opportunity to revise our manuscript, ``Multiphysics Reactor-core Simulations Using the Improved Quasi-Static Method''. We appreciate the prompt review and constructive suggestions. It is our belief that the manuscript is substantially improved after making the suggested edits.

Overall, the comments from the reviewer were positive and pivotal in the improved quality of the manuscript. Following this letter are the review comments in italics with our responses below, including how and where the text was modified.

We hope you will find the revised manuscript suitable for publication. Thank you again for considering this manuscript for publication in the {\it Annals of Nuclear Energy}.

\vspace{0.25cm}


%%%%%% More vertical space can be added here %%%%%%%%%%%%%%%%%%%%%%%%%%%%%%

%%%%%%% The Closing %%%%%%%%%%%%%%%%%%%%%%%%%%%%%%%%%%%%%%%%%%%%%%%%%%%%%%%
\closing{Sincerely, }

\end{letter}

\large{\textbf{Reviewer Comments}}\hrule\small

\section{Major Issues}

\begin{itemize}

\item[\done] \medm{ Justify page 8: ``we expect temperature to be more rapidly varying than the shape but less so than the amplitude" }
\begin{itemize}
\item We have added the following on Page 9: ``This idea stems from the fact that in an adiabatic temperature model, the temperature's spatial variation follows the neutron flux shape, while the temperature amplitude follows the flux amplitude. Because of the heat capacity of the fuel, the temperature amplitude is often more slowly varying in time than neutron amplitude function itself, while still considerably faster varying than the flux shape.'' This conjecture is empirically defended in the results of Section 5, for instance, the new Figure 17.
\end{itemize}

\item[\done] \easy{ Justify page 5: ``Using IQS, one expects the time dependence of the shape to be weaker than that of the flux itself, this allowing for larger time step sizes in updating the shape". }
\begin{itemize}
\item Page 5: References [1] as justification for this statement and maybe its first empirical numerical verification. However, since K. Ott, many other researchers used and published IQS results. Our literature reviews includes several of these references, some recent, more older.
\end{itemize}

\item[\working] \medm{  Results are often just stated and not discussed, e.g. page 19 ``It also shows that traditional IQS performed better with large etol, while \iqspc was better with smaller etol". The statement simply bloats the manuscripts; discuss the reasons and implications or remove it. Another example is page 15, ``However, switching the precursors solve to be performed semi-analytically…".  In this case, the authors should take the time to discuss the implications of this finding. }

Thank you for the recommendations. The following sentences have been added:
\begin{itemize}
\item Page 16: Added ``This result implies that uniqueness of the shape and amplitude can only be maintained with a consistent treatment of precursor term in the shape equation."
\item Page 18: Removed ``This illustrates that IQS can yield a much more accurate solution, even at a significantly larger time step than the implicit flux discretization."
\item Page 20: Added ``These convergence results are consistent with the hypothesis that the performance of IQS is highly dependent on the spatial coupling of the flux."
\item Page 20: Added ``This observation implies that the time adaptive options must be chosen differently when applied to a solver for shape versus a solver for flux."
\end{itemize}

\item[\done] \medm{ For the comparison of LRA and TREAT results, the efficiency [how accurate can you be given a fixed execution time or given an accuracy how much execution time do you need] is a very important quantity; the authors discuss it for LRA and TREAT. However, the authors first present results for the errors and then runtime results. Then they try to synthesize efficiency by comparing different cases that have different runtimes and different errors, e.g. page 23 right before Table 5 ``\iqspc shows and error less than implicit discretization at delta t = 0.002, signifying …". I suggest the authors create error vs. execution time plots for different cases and based on these curves discuss efficiency. The manner efficiency is discussed is very confusing and not very useful for the reader. }
\begin{itemize}
\item Thank you for the suggestion. We have added this on Page 24: ``Fig. 17 visualizes the dependence of run-time on error. This figure shows that IQS and IQS P-C generally has a lower run-time for certain error than implicit discretization. Furthermore, note that the slopes on the right half of the \iqspc runs are shallower than the implicit discretization slope. This shows that by adding more temperature updates, in the 1-4 temperature-updates range, \iqspc suffers a lesser increase in run-time for a decrease in error. However, beyond 4 temperature updates, the error in the shape function saturates and no additional benefits can be gained by more frequent temperature updates. "
\end{itemize}

\item[\done] \easy{ In the results section, Error is sometimes used without introducing it, e.g. page 25, ``Table 9 shows the error ….". Further ``Error" is often used as y-label in plots., e.g. in the discussion of Figure 10 in which ``convergence results" are presented; the sentence is not clear in this context because it was not stated if this is iterative convergence or convergence to the right solution as the time step is reduced. In some instances error is defined but it is tedious for the reader to find the corresponding passage in the text, figures, or tables. }
\begin{itemize}
\item On  Pages 17-25: We included an explicit definition of error in each of the plots and tables. Thank you for the suggestion.
\end{itemize}

\item[\done] \hard{ LRA and TREAT only compare errors computed with the peak powers. Peak powers are an integral quantity and may favor IQS over spatial kinetics. The authors should also compare power distributions or similar distributed quantities. }
\begin{itemize}
\item Thank you for the suggestion. On Page 24-26, we added: ``Up to now, errors presented in this section were norms (i.e., space-integrated flux errors). However, for reactor physics problems, it is also noteworthy to compare the spatial dependence of error, for example, error in the power distribution. Fig. 18 shows the difference in power for each assembly of the LRA geometry for each method. The resulting spatially dependent error is in general an order of magnitude lower for IQS and \iqspc than for the implicit flux discretization. This observation agrees well with the integrated flux errors shown in Tables 6-8. "
\end{itemize}

\item[\done] \medm{  Whenever errors are computed, baseline calculations are performed. How do the authors know that these baseline calculations are accurate enough to measure the error w.r.t. them? }

Thank you for suggesting this clarification. We've made the following modifications for justification:
\begin{itemize}
\item On Page 18: we added ``It should be noted that the baseline for these error computations was performed using an adaptive method for the flux equations with a tight tolerance. The proper convergence rates seen by the methods with coarser time steps show that this baseline is accurate enough for computing the error."
\item On Page 20: we added ``The baseline was computed using the BDF2 scheme on the flux equation with a time step of $10^{-5}$. The proper error convergences show that this solution is accurate enough to compare to for error computation for most of the convergence data points. Smallest time step scheme for IQS shows a break in the linearity of the convergence."
\end{itemize}

\end{itemize}

\newpage
\section{Minor Issues}

\begin{itemize}

\item[\done] \easy{ Page 1: ``… the amplitude, and a space- and time-dependent component, the shape [1,2,3,4,5]". What about energy? }
\begin{itemize}
\item We added this sentence on Page 1: ``IQS involves factorizing the neutron flux solution into a time-only-dependent component, the amplitude, and a full phase-space component, the shape."
\end{itemize}

\item[\done] \easy{ Is $\phi^g$ the scalar flux or scalar flux moments in Eq 1a? }
\begin{itemize}
\item Thank you for catching this discrepancy. On Page 3 we've added: ``Note that $\phi^g$ for neutron transport is the angular neutron flux ($\phi^g(\vec r, \vec\Omega, t)$), while for neutron diffusion it is scalar flux ($\phi^g(\vec r, t)$)."
\end{itemize}

\item[\working] \easy{ All quantities in Eq. 1a and 1b need to be defined. }
\begin{itemize}
\item We agree. On Page 4: Created a table of the operator definitions (Table 1)
\end{itemize}

\item[\done] \easy{ Eq. 3a: $S_d^g$ is not defined. }
\begin{itemize}
\item We agree. On Page 4: Added in Table 1.
\end{itemize}

\item[\done] \easy{ For the uninitiated reader the term phase-space domain may not be familiar. The authors could introduce the term phase space. IQS may be interesting for people in other fields. }
\begin{itemize}
\item We agree. On Page 1, we added: ``In neutron transport, the neutron flux solution lives in a seven dimensional phase-space, dependent on time, space, energy, and direction. The neutron diffusion approximation reduces this phase-space by eliminating direction."
\end{itemize}

\item[\done] \easy{ Algorithm in section 2.1 on page 5. Step 2: ``linearly interpolate". It is later stated that interpolation of higher order is done for high order schemes. }
\begin{itemize}
\item Thank you for catching this. Indeed, we use high-order interpolation when a higher-order scheme is used. We removed ``linearly'' from step 2 on Page 5.
\item We've also add on Page 6: ``It should be noted that Step 2 of the process generally involves a linear interpolation; however, for greater than second order shape discretizations, a higher-order interpolation is necessary."
\end{itemize}

\item[\nofix] \easy{ For the Linf nor of the shape: why is the maximum difference over the maximum value used and not the maximum relative difference ($1 - \phi^{k+1}/\phi^k$). }
\begin{itemize}
\item We included the reference for this error estimation.
\end{itemize}

\item[\done] \medm{ Eq (11) is restated in Eq (24). Why not just make Eq. 11 more general. Also, in Eq. (24) cp(T(t)) and originally the heat equation reads d(rho * cp(T) * T). Hence, the cp(T) cannot be pulled out so Eq. (24) is incorrect. }
\begin{itemize}
\item Whether the heat capacity depends on temperature or not, the term  $\rho C_p(T)$ should always be outside of the temporal derivative. This form is given, for instance, in the Todreas \& Kazimi textbook ``Nuclear Systems I: Thermal-hydraulic fundamentals'', as Equation 4-123.
\end{itemize}

\item[\done] \easy{ Figure 2. uses the delta t / 3 step that was claimed to be only for illustration purposes in Fig. 1. }
\begin{itemize}
\item Page 10: We included the variable $N_{T}$ in Figure 2 for clarity.
\item $N_T$ is defined on Page 9: ``$N_T$ is the number of temperature updates per macro step.''
\end{itemize}

\item[\done] \medm{ Statement page 9: ``Furthermore, a very accurate representation of p(t) over the macro step is available from the PRKE solve…". I think it should be precise or else one would have to show that the equations are solved accurately. Second, what does ``very accurate" mean, ``very accurate" is comparison to what? }
\begin{itemize}
\item Thank you for noting this. We improved the phrasing on Page 11: ``Furthermore, a highly refined representation of $p(t)$ over the macro step is available from the PRKE solve (micro time scale). Using this fine scale information yields a more precise integration of amplitude than assuming an interpolation of the amplitude between macro step points."
\end{itemize}

\item[\done] \medm{ Implementation uses code specific jargon: kernels, auxkernels, Transient executioner, user-object. These need to be explained or cast in more general terms. }
\begin{itemize}
\item Thank you for the valuable suggestion. This has been added on Page 13:
\begin{itemize}
\item Kernel - Evaluation of the weak-form residual for a particular piece of physics.
\item Executioner - Establishes the type of simulation. The Transient type moves the simulation in time conducting spatial evaluations at each step.
\item Auxiliary variable and kernel - ``Optional" variables that live on the same mesh as the solution and are computed algebraically using auxiliary kernels. 
\item Postprocessor - Computes scalar values over the entire spatial mesh, usually involves integrated quantities.
\item User-object - A generic type of postprocessor that allows connectivity of relevant quantities between different MOOSE objects.
A more detailed and comprehensive description of MOOSE objects can be found in [18]
\end{itemize}
\end{itemize}

\item[\done] \easy{ Page 13 top of the page ``looping over cells to evaluate", what cells, mesh cells? }
\begin{itemize}
\item Thank you for suggesting this clarification. Page 14: ``The PRKE parameters are written as user-objects, looping over spatial mesh cells to evaluate them."
\end{itemize}

\item[\done] \easy{ LRA benchmark: the number of elements makes no sense 11x11 = 121 and when using five uniform refinement steps one would end up with 123,904 elements. }
\begin{itemize}
\item Thank you, that number is correct. On Page 22: Fixed it to 123,904 elements, number of nodes was originally correct. 
\end{itemize}

\item[\done] \easy{ LRA benchmark description does not state what equations are solved, if that is provided in the ANL benchmark book it should be made clear. Equations could also be stated directly. }
\begin{itemize}
\item  Page 22: ``The equations being solved are explicitly defined in problem 14-A1 of the ANL Benchmark Problem Book [20]"
\end{itemize}

\item[\done] \easy{ In plots where delta t is plotted on the x-axis, it should be made clear that it is the macroscopic time step. }
\begin{itemize}
\item On Page 14 and 22, we added: ``Note that any instance of $\Delta t$ in the following figures and commentary signifies the size of the macro time step."
\end{itemize}

\item[\done] \easy{ Conclusions, page 27: incomplete sentence ``IQS showed expected error convergence up". }
\begin{itemize}
\item Thank you. Page 31: ``...through fourth-order time discretization of the shape"
\end{itemize}

\item[\nofix] \easy{ Timing estimates for LRA: the author should make clear how they execute the LRA problem, i.e. number of processors and processor model. Given the long runtimes, it appears that a single processor is used. Is this to avoid problems with asynchronous behavior in parallel? }
\begin{itemize}
\item Already stated on Page 24: ``These run-times are based on total alive time of the execution where the diffusion evaluation is distributed over 24 processors."
\end{itemize}

\end{itemize}

\section{Bibliography}

\begin{itemize}

\item[\done] \easy{ MOOSE reference is out of date. Recommend newer publication }
\begin{itemize}
\item Thank you. We replaced it with [16].
\end{itemize}

\item[\done] \easy{ Rattlesnake reference [17] is out of date. Recommend citing the theory manual INL/EXT-17-42103. }
\begin{itemize}
\item Thank you. We replaced it with [17].
\end{itemize}

\item[\done] \easy{ The LRA benchmark is extensively discussed in the Rattlesnake user manual, which might be a handy reference to use: INL/EXT-15-37337. }
\begin{itemize}
\item Thank you for suggesting to include this reference. We've included it on Page 22: ``The execution of the benchmark was performed by the Rattlesnake/MOOSE framework at Idaho National Laboratory (INL) [25]"
\end{itemize}

\end{itemize}


\end{document}