%%%%%%%%%%%%%%%%%%%%%%%%%%%%%%%%%%%%%%%%%%%%%%%%%%%
%
%  New template code for TAMU Theses and Dissertations starting Fall 2016.  
%
%  Author: Sean Zachary Roberson
%	 Version 3.16.09
%  Last updated 9/12/2016
%
%%%%%%%%%%%%%%%%%%%%%%%%%%%%%%%%%%%%%%%%%%%%%%%%%%%
%%%%%%%%%%%%%%%%%%%%%%%%%%%%%%%%%%%%%%%%%%%%%%%%%%%%%%%%%%%%%%%%%%%%%
%%                           ABSTRACT 
%%%%%%%%%%%%%%%%%%%%%%%%%%%%%%%%%%%%%%%%%%%%%%%%%%%%%%%%%%%%%%%%%%%%%

\chapter*{ABSTRACT}
\addcontentsline{toc}{chapter}{ABSTRACT} % Needs to be set to part, so the TOC doesnt add 'CHAPTER ' prefix in the TOC.

\pagestyle{plain} % No headers, just page numbers
\pagenumbering{roman} % Roman numerals
\setcounter{page}{2}

\indent Transient reactor simulation is a chronically formidable challenge due to the computational rigor of evaluating the neutron transport equation, as well as its coupling with other physical phenomena. The goal of this thesis research is to mitigate the computational expense of these simulations by investigating and developing the improved quasi-static method (IQS). IQS involves factorizing neutron flux within a reactor into a time-dependent amplitude and time-space-energy dependent shape.  The purpose is to evaluate amplitude and shape on different time scales in order to reduce computational expense, while maintaining solution accuracy. IQS factorization leads to a nonlinear system of equations that requires iteration of shape and amplitude. Additionally, reactor simulation requires multiphysics simulation, where the quasi-static methodology can also be applied.

The objectives of this research is to establish IQS performance with various iteration techniques, validate time step convergence of IQS, and apply IQS to multiphysics simulation. IQS iteration techniques involve fixed-point (Picard) iteration with various convergence criteria and Newton iteration, namely preconditioned Jacobian-free Newton Krylov (PJFNK) method. Nonlinear convergence of each of these techniques is investigated. Validation of IQS with analysis of time step convergence is vital for implementation of time adaptive methods and error prediction. The time derivative of the shape function is discretized through fourth order using implicit-Euler, Crank-Nicolson, backward difference formula (BDF), and singly-diagonally-implicit Runge-Kutta (SDIRK) methods. IQS application to multiphysics simulations involves its implementation into the Rattlesnake/MOOSE framework. These simulations allow insight to the performance of IQS for full transient reactor simulations.

The results of the iteration convergence analysis show that the most rigorous and comprehensive iteration technique is fixed-point iteration with consistency in the IQS uniqueness specification as the convergence criteria. This iteration technique revealed the need for analytical treatment of the precursor equation for proper convergence. For time step convergence analysis, IQS was applied to a one-dimensional prototype example, as well as the TWIGL and LRA benchmark.  The prototype results show that IQS has proper error convergence through fourth order discretization schemes. The TWIGL and LRA benchmark results show that IQS has proper convergence for implicit Euler, second-order BDF, and Crank-Nicolson schemes, validating IQS for more complex problems. For multiphysics simulation, IQS was applied to the LRA benchmark and a full core TREAT model. The results show that integration of temperature into the quasi-static process made considerable improvement to IQS performance. These results and conclusions helped gain insight into the behavior of IQS and furthered its development in complete transient reactor models.


 

\pagebreak{}
