%%%%%%%%%%%%%%%%%%%%%%%%%%%%%%%%%%%%%%%%%%%%%%%%%%%
%
%  New template code for TAMU Theses and Dissertations starting Fall 2016.
%
%  Author: Sean Zachary Roberson 
%	 Version 3.08.16
%  Last updated 8/19/2016
%
%%%%%%%%%%%%%%%%%%%%%%%%%%%%%%%%%%%%%%%%%%%%%%%%%%%

%%%%%%%%%%%%%%%%%%%%%%%%%%%%%%%%%%%%%%%%%%%%%%%%%%%%%%%%%%%%%%%%%%%%%%
%%                           SECTION I
%%%%%%%%%%%%%%%%%%%%%%%%%%%%%%%%%%%%%%%%%%%%%%%%%%%%%%%%%%%%%%%%%%%%%


\pagestyle{plain} % No headers, just page numbers
\pagenumbering{arabic} % Arabic numerals
\setcounter{page}{1}


%%%%%%%%%%%%%%%%%%%%%%%%%%%%%%%%%%%%%%%%%%%%%%%%%%%%%%%%%%%%%%%%%%%%%
%%%%%%%%%%%%%%%%%%%%%%%%%%%%%%%%%%%%%%%%%%%%%%%%%%%%%%%%%%%%%%%%%%%%%
\chapter{\uppercase{Introduction}}
%%%%%%%%%%%%%%%%%%%%%%%%%%%%%%%%%%%%%%%%%%%%%%%%%%%%%%%%%%%%%%%%%%%%%
%%%%%%%%%%%%%%%%%%%%%%%%%%%%%%%%%%%%%%%%%%%%%%%%%%%%%%%%%%%%%%%%%%%%%

%%%%%%%%%%%%%%%%%%%%%%%%%%%%%%%%%%%%%%%%%%%%%%%%%%%%%%%%%%%%%%%%%%%%%
\section{Background on Transient Reactor Testing}
%%%%%%%%%%%%%%%%%%%%%%%%%%%%%%%%%%%%%%%%%%%%%%%%%%%%%%%%%%%%%%%%%%%%%

The primary purpose of transient reactor testing is the safety and performance analysis of fuel and other reactor components. Since the March 2011 events at the Fukushima Daiichi Nuclear Power Plant in Japan, there has been significant governmental demand for the development of accident tolerant fuels in light water reactors (LWR) \cite{ConRepTREAT}.  Currently, two facilities are capable of testing fuels under these extreme conditions: the Annular Core Research Reactor (ACRR) at Sandia National Laboratory (SNL) and the Transient Reactor Testing Facility (TREAT) at Idaho National Laboratory (INL) \cite{SandiaTREAT}. TREAT had been put in stand-by status in 1994, but in 2014 a Final Environmental Assessment by the Department of Energy (DOE) approved the restart of the facility in order to resume nuclear fuel testing\cite{FONSI}. 

TREAT is an graphite-moderated, air-cooled, thermal reactor which began operation in 1959.  The reactor was designed to subject fuels and experimental apparatus to extreme power pulses, for fast reactor type components \cite{TREATSummary}. The primary purpose for the restart of the facility is to use these pulses to simulate accident scenarios and test the integrity of accident tolerant LWR fuels. TREAT is expected to resume experimentation and testing by 2020.  In addition to the substantial upgrades of TREAT's electronic and mechanical systems during the restart, advance computer models of the reactor are in development.  These models have a mutualistic relationship with TREAT: validating fuel models with experimentation and using reactor models to streamline experimental procedures. Development of these models has renewed a significant interest in implementing transient reactor simulation methods in modern computational tools.

Transient reactor simulation methods have been developed for several decades. The revitalization of TREAT and other transient reactor testing efforts have brought special attention to the methods' implementation in modern computational tools.  The DOE, through its NEAMS (Nuclear Energy Advanced Modeling and Simulation) program, has especially sought the development of transient multiphysics capability in INL's MOOSE (Multiphysics Object-Oriented Simulation Environment). This research intends to investigate and develop the improved quasi-static method (IQS) with these modern tools. IQS is a transient neutronics method intended to improve computational efficiency. Improving this efficiency is vital for reactor simulations that would otherwise be overbearing to evaluate. In order to apply IQS to full transient reactor models, implementation of multiphysics is required. The following two sections provide an overview of time-dependent neutronics and multiphysics solution methods. 

%%%%%%%%%%%%%%%%%%%%%%%%%%%%%%%%%%%%%%%%%%%%%%%%%%%%%%%%%%%%%%%%%%%%%
\section{Time-Dependent Neutron Transport}
%%%%%%%%%%%%%%%%%%%%%%%%%%%%%%%%%%%%%%%%%%%%%%%%%%%%%%%%%%%%%%%%%%%%%

Transient modeling of nuclear reactors has been a chronically formidable task due to its computationally expensive nature. The difficulty in transient reactor modeling is due to:
\begin{enumerate}
\item
the high-dimensionality of the phase-space for the governing equations that describe the flux of neutrons (6-D+time for multigroup neutron transport and 4-D+time for multigroup neutron diffusion), and 
\item
the fact that the time discretization has to be implicit, which  leads to a stiff system of equations.  
\end{enumerate}
The transport equation has seven independent variables: space ($\vec{r}$), energy ($E$), direction ($\vec{\Omega}$), and time ($t$) \cite{duderstadt1976nuclear}.  Nuclear reactor simulations often utilize the neutron diffusion approximation, which carefully eliminates the dependence on direction.  For the purpose of this research, neutron diffusion is discussed exclusively.

For computational purposes, the neutron diffusion equation is discretized in space, energy, and time.  There are several viable discretization schemes for each of these variables. In this research effort, FEM is used for space discretization and multigroup for energy discretization \cite{zeinkiewicz2005finite, duderstadt1976nuclear}. Since the purpose of this research involves temporal dependence, discussion of tempoeral discretization is kept general at this stage. \tcr{ok with you?} Most reactor computation frameworks discretize the time variable of flux directly using a multitude of schemes (Euler, Crank-Nicholson, Runge-Kutta, etc.).  However, the solutions of flux can be particullary stiff due to the speed of neutrons, especially in a pulsing reactor \cite{TWIGL_benchmark}. Therefore, computationally expensive implicit schemes are necessary with many points of discretization (time steps). The following subsections describe two alternate techniques for transient simulation: point reactor kinetics and the improved quasi-static method (IQS). These methods are meant to reduce computational expense of evaluating time-dependent neutron population distributions while maintaining accuracy.

%%%%%%%%%%%%%%%%%%%%%%%%%%%%%%%%%%%%%%%%%%%%%%%%%%%%%%%%%%%%%%%%%%%%%
\subsection{Point Reactor Kinetics}

A common solution process for neutron kinetics is the evaluation of the point reactor kinetics equations (PRKE) \cite{Planchard_1991}.  The derivation of this equation involves factorizing flux into a space-dependent shape and time-dependent amplitude.  This factorization immediately creates a large assumption that the spatial variance of the flux is time-independent.  The amplitude has the same temporal stiffness as flux; however, the PRKE takes very little computational effort to evaluate because the many variables from the spatial discretization only need to be evaluated for the initital (steady state) distribution. For all times into the transient, the spatial shape is assumed fixed and one just need to evaluate the PRKE, a small system of ODEs,for the time-dependent amplitude.


%%%%%%%%%%%%%%%%%%%%%%%%%%%%%%%%%%%%%%%%%%%%%%%%%%%%%%%%%%%%%%%%%%%%%
\subsection{Improved Quasi-Static Method}

IQS is a spatial kinetics method that involves factorizing the flux solution into space- and time-dependent components \cite{Ott_1966,Dulla2008,Devooght_1984,Monier_diss,Sissaoui_1995}.  These components are the flux's amplitude and its shape. Amplitude is only time-dependent, while the shape is both space- and time-dependent (as opposed to the PRKE approach where the shape was constant in time).  However, the impetus of the method is the assumption that the shape is only weakly dependent on time; therefore, the variable for the multi-D diffusion solves is expected to be less stiff than the flux itself, without making the assumption that spatial variance is time-independent.  As a result, the shape may not require an update at the same frequency as the amplitude function, but only on larger macro-time steps.  

Due to the factorization of the flux into a shape and an amplitude, the latter two variables are coupled. Ott in \cite{Ott_1966} first investigated the coupling of shape and amplitude in a quasi-static nature, but did not include the time derivative in the shape equation. Later, in \cite{Ott_1969}, Ott incorporated the time derivative of shape in the equation, yielding better results; this also led to the technique's name: the Improved Quasi-Static method.  The resulting system of equations is  nonlinear \cite{Dulla2008}. Nonlinear problems require an iterative process to evaluate, either fixed-point (Picard) or Newton iteration. Sissaoui et al. \cite{Sissaoui_1995}, Koclas et al. \cite{Koclas_1996}, Devooght et al. \cite{Devooght_1984}, and Monier \cite{Monier_diss} all use fixed-point iterative techniques for their IQS simulations, the main difference among them is their criteria for convergence.  Devooght et al. in \cite{Devooght_1984} also utilizes a Newton iteration technique.

%%%%%%%%%%%%%%%%%%%%%%%%%%%%%%%%%%%%%%%%%%%%%%%%%%%%%%%%%%%%%%%%%%%%%
\section{Time-Dependent Reactor Dynamics}
%%%%%%%%%%%%%%%%%%%%%%%%%%%%%%%%%%%%%%%%%%%%%%%%%%%%%%%%%%%%%%%%%%%%%

Transient reactor analysis  involves coupling neutronics with other physics such as thermal hydraulics, fuel mechanics, etc. to to enable temperature feedback in the neutronics simulations. This coupling is known as multiphysics simulation. Multiphysics simulation is often a daunting task because it requires communication of coupled variables that are being evaluated with a myriad of techniques. The coupling is also usually nonlinear requiring Picard or Newton iteration techniques. Programs, such as MOOSE, provide the capability of evaluating, communicating, and iterating different physics. 

MOOSE is a multiphysics framework being developed at INL that presents the architecture for physics-based applications \cite{moose}.  In order to robustly implement IQS in the MOOSE framework for multiphysics simulation, each coupled physics application requires consideration. Rattlesnake is a large application in MOOSE that involves deterministic radiation transport physics.  Currently, Rattlesnake is able to solve steady-state, transient, and k-eigenvalue neutron transport and diffusion problems using finite element methods (FEM) \cite{wang2013}. BISON is the application for evaluating fuel temperature and structural properties \cite{bison}.  RELAP-7 is the main safety analysis application, but focuses on variables involved with hydrodynamic analysis \cite{relap7}.  MAMMOTH is the over-arching application that couples all these physics by recomputing cross-sections for Rattlesnake and source terms for BISON and RELAP-7 \cite{mammoth}.

IQS is a manipulation of neutronics variables, so it is solely developed within the Rattlesnake application. However, proper treatment of multiphysics with IQS is vital for computational optimization.  The quasi-static nature of IQS can be extended to multiphysics simulation by determining the proper time scales for each physics.  Considering TREAT, shape is the slowest varying variable and amplitude is fastest, while fuel temperature is in between.  Varying time step sizes for each variable can considerably improve computational efficiency, which is the nature and purpose of IQS application.

%%%%%%%%%%%%%%%%%%%%%%%%%%%%%%%%%%%%%%%%%%%%%%%%%%%%%%%%%%%%%%%%%%%%%
\section{Objective}
%%%%%%%%%%%%%%%%%%%%%%%%%%%%%%%%%%%%%%%%%%%%%%%%%%%%%%%%%%%%%%%%%%%%%

The goal of this research is to continue the investigation and development of the improved quasi-static method for minimizing computation expense for transient reactor simulations, while maintaining accuracy. This goal is pursued with three objectives:

\begin{enumerate}
\item Establish IQS performance for various iteration techniques
\item Validate time step convergence for IQS with high order time discretization schemes
\item Apply IQS to multiphysics simulations
\end{enumerate}

Iteration convergence analysis involves investigating iteration techniques for nonlinear problems and convergence criteria. For this objective, IQS is implemented into an easily modifiable, prototype-like code in MATLAB. High order time discretizing schemes were also implemented in the prototyping code to investigate time step convergence. The prototype also guides the development of IQS in the large FEM framework of Rattlesnake/MOOSE, where more complex examples and benchmarks can be tested.  Time step convergence is also analyzed for these examples. Development in Rattlesnake also spawns the opportunity to test IQS with multiphysics, namely temperature feedback and time adaptation. Applying IQS to multiphysics simulations allows insight to the performance of IQS for full transient reactor simulations, particularly for TREAT modeling.

%%%%%%%%%%%%%%%%%%%%%%%%%%%%%%%%%%%%%%%%%%%%%%%%%%%%%%%%%%%%%%%%%%%%%%%%%%%%%%%%%%%%%%%%%%%%%%%%%%%%%

