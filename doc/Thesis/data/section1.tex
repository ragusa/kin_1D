%%%%%%%%%%%%%%%%%%%%%%%%%%%%%%%%%%%%%%%%%%%%%%%%%%%
%
%  New template code for TAMU Theses and Dissertations starting Fall 2016.
%
%  Author: Sean Zachary Roberson 
%	 Version 3.08.16
%  Last updated 8/19/2016
%
%%%%%%%%%%%%%%%%%%%%%%%%%%%%%%%%%%%%%%%%%%%%%%%%%%%

%%%%%%%%%%%%%%%%%%%%%%%%%%%%%%%%%%%%%%%%%%%%%%%%%%%%%%%%%%%%%%%%%%%%%%
%%                           SECTION I
%%%%%%%%%%%%%%%%%%%%%%%%%%%%%%%%%%%%%%%%%%%%%%%%%%%%%%%%%%%%%%%%%%%%%


\pagestyle{plain} % No headers, just page numbers
\pagenumbering{arabic} % Arabic numerals
\setcounter{page}{1}


\chapter{\uppercase{Introduction}}

\tcr{Advice: rename through git the files section1 ... with more meaningful filenames, e.g., introduction, theory, ...}

\tcr{Advice: use bibtex for bibliography in the future, rather than bibitem, so that you can change the bibliography
style super quickly. Also, most journals provide the bibtex entry for the article, so you just save that in your bib file - yes, no need to type much if you use that feature-}

Transient modeling of nuclear reactors has been a chronically foreboding task due to its computationally expensive nature.  However, recent developments in nuclear testing
\tcr{this is not the introduction of a short summary, you can spend a bit more time in the topic of transient testing,
what is it for, what facilities, how what it simulated before, why a regain of interest in simulation for this today, ... build the storyline...}, including the revitalization of the Transient Reactor Testing (TREAT) Facility at Idaho National Laboratory (INL), have brought significant attention to the development of transient reactor modeling.  3-D transient computational methods currently in use at national laboratories, including INL, have proven to be overbearingly computationally expensive for complex, real-world scenarios.  Therefore, implementation of more efficient and faster methods for 3-D transient simulations is highly desired \tcr{this was already a problem when Karl Ott introduced IQS in 1977(?) so to rationale sounds a bit hollow the way you present it and yet we are doing it and Downar was awarded last year a grant to do this too. I think the real reason is that most modern reactor physics tools have not being used much for fast transients, so the only thing we are doing is bringing back the method into modern tools. However, in doing so, }.  The Department of Energy (DOE), through its NEAMS (Nuclear Energy Advanced Modeling and Simulation) program, has especially sought the development of transient multiphysics capability in INL's MOOSE (Multiphysics Object-Oriented Simulation Environment).  \\

\tcr{
I think an outline for the introduction should be 
\begin{enumerate}
\item reactor transient testing. Background, etc... this subsection should conclude on the overarching need of transient testing: computational methods for multiphysics reactor transients. It should also introduce the next 2 subsections, which are both about lit review, one on time-dependent  neutronics, one on multiphysics solution techniques for reactor physics.
\item literature review on mostly neutronics time dependent: challenges in computational methods for time-dependent neutron balance equations. the size of the phase-space, the need for implicitness in the temporal treatment due to the stiffness, .... this should open up to a discussion on alternate methods, such as IQS, ... it is a lit review, so no need for equations, but rather a demonstration that you know the background and have a solid understand of that foundation to lay the ground from your OWN work. As part of this lit, some discussion of Mund's, Dulla's, Ott's, Monier,'s the Canadian's papers is welcome. 
\item now, reactor transients also means coupled to other physics. this is were you discuss some aspect related to multiphysics. you can start with the general discussion that this is typically a nonlinear problem, that it can be solved via picard or newton. this is maybe a segway into a discussion/introduction about MOOSE. Then you go back on IQS and add the multiphysics aspects. Ott did some work with multiphysics, I gave you the code manual of his own software where he discusses when he decided to do shape updates.
\end{enumerate}
Generally speaking, the rest is real good but you need to spend some time on these 5-8 pages of introduction that set the tone and the background. Everybody needs to understand that you know a lot; that your thoughts are well organized and clear; the rest will come easy (except some results ...).\\
%
side note: we need to put the small results about sdirk being poor for stiffness problem. maybe as an appendix. I want to keep that knowledge somewhere. 
}
MOOSE is a multiphysics framework being developed at INL that presents the architecture for physics-based applications \cite{moose}.  Rattlesnake is a large application in MOOSE that involves deterministic radiation transport physics.  Currently, Rattlesnake is able to solve steady-state, transient, and k-eigenvalue neutron transport and diffusion problems using finite element methods (FEM) \cite{wang2013}. \\

The difficulty in transient reactor modeling is due to 
\begin{enumerate}
\item
the high-dimensionality of the phase-space for the governing equations that describe the flux of neutrons (6-D+time for multigroup neutron transport and 4-D+time for multigroup neutron diffusion), and 
\item
the fact that the time discretization has to be implicit, which  leads to stiff system of equations.  
\end{enumerate}
The transport equation has seven independent variables: space ($\vec{r}$), energy ($E$), direction ($\vec{\Omega}$), and time ($t$) \cite{duderstadt1976nuclear}.  Nuclear reactor simulations often utilize the neutron diffusion approximation, which carefully eliminates the dependence on direction.  For the purpose of this research, neutron diffusion will be discussed exclusively. \\

\section{Neutron Diffusion Solution Process}

For computational purposes, the neutron diffusion equation is discretized in space, energy, and time.  There are several viable discretization schemes for each of these variables; FEM for space and multigroup in energy are used for the development of this research. Since the purpose of this research involves temporal dependence, discretization in time was kept general.  The time dependent neutron diffusion equation with delayed neutron precursors can be seen in Equations~\ref{eq:flux}~and~\ref{eq:precursor}.

\begin{subequations}
\begin{multline}
\frac{1}{v^g}\frac{\partial \phi^g}{\partial t}  = 
\frac{\chi_p^g}{\keff} (1-\beta)\sum_{g'=1}^G  \nu^{g'} \Sigma_f^{g'} \phi^{g'} 
+ \sum_{g'\neq g}^G\Sigma_s^{g'\to g} \phi^{g'}  \\ + \sum_{i=1}^I\chi_{d,i}^g\lambda_i C_i 
-  \left( -\div D^g \grad  + \Sigma_r^g \right) \phi^g   
\ , \quad 1 \le g \le G 
\label{eq:flux}
\end{multline}
\be
\frac{dC_i}{dt} = \frac{\beta_i}{\keff}\sum_{g=1}^G\nu^{g} \Sigma_f^g \phi^{g} - \lambda_i C_i \ , \quad 1 \le i \le I 
\label{eq:precursor}
\ee
\end{subequations}

where,

\begin{tabular}{lll}
$\phi^g$   			&	$=$	&	Scalar flux in energy group $g$ \\
$C_i$					  &	$=$	&	Concentration of delayed neutron precursor $i$ \\
$\Sigma_f^{g}$	&	$=$	&	Fission cross section in energy group $g$ \\
$\Sigma_r^{g}$	&	$=$	&	Removal cross section in energy group $g$ \\
$\Sigma_s^{g' \to g}$	&	$=$	&	Scattering cross section from energy group $g'$ to $g$ \\
$v^g$					  &	$=$	&	Neutron velocity in energy group $g$ \\
$\chi_p^g$			&	$=$	&	Fission spectrum of prompt neutrons \\
$\chi_{d,i}^g$	&	$=$	&	Fission spectrum of delayed neutrons from precursor $i$ \\
$\nu^g$					&	$=$	&	Total number of neutrons per fission \\
$D^g$					  &	$=$	&	Diffusion coefficient in energy group $g$\\
$\lambda_i$			&	$=$	&	Decay constant of precursor $i$ \\
$\beta_i$				&	$=$	&	Delayed neutron fraction  from precursor $i$ \\
$\beta$			 	  &	$=$	&	Total delayed neutron fraction ($\beta = \sum_{i=1}^I \beta_{i}$) \\
  & & 
\end{tabular}

Most reactor computation frameworks, including Rattlesnake, discretize the time variable directly with these equations using a multitude of schemes (Implicit Euler, Crank-Nicholson, implicit Runge-Kutta, etc.).  In this paper, the method of discretizing Equations~\ref{eq:flux}~and~\ref{eq:precursor} is generally referred to as ``implicit discretization''.  This research intends to improve upon this method by instead implementing the improved quasi-static method (IQS) for neutron kinetics and
implement it within a multiphysics setting. \\

\subsection{Point Reactor Kinetics Equation}

A common solution process for neutron kinetics is the evaluation of the point reactor kinetics equation (PRKE).  The derivation of this equation involves factorization flux into a space-dependent shape and time-dependent amplitude.  This factorization immediately creates a large assumption that the spacial variance of the flux is time-independent.  However, this takes very little computational effort to evaluate because the multitudinous variables from the spacial discretization only needs to be evaluated on the initial time step.  The rest of the time steps only evaluated one variable: amplitude.

\subsection{Improved Quasi-Static Method}

IQS is a spatial kinetics method that involves factorizing the flux solution into space- and time-dependent components \cite{Ott_1966,Dulla2008}.  These components are the flux's amplitude and its shape. Amplitude is only time-dependent, while the shape is both space- and time-dependent.  However, the impetus of the method is the assumption that the shape is only weakly dependent on time; therefore, the shape may not require an update at the same frequency as the amplitude function, but only on larger macro-time steps. \\
\indent Since shape and amplitude share a independent variable, they are temporally coupled. Additionally, the resulting system of equations is heavily nonlinear.  The are a number of published strategies that evaluate this system, but a in depth comparison of the techniques has not been found.  

\section{Objective}

The goal of this research is to continue the investigation and development of the improved quasi-static method.  This goal requires the development of IQS in a easily modifiable, prototype-like code, as well as a robust FEM framework. \\
\indent The prototyping code is used to investigate time discretization and nonlinear iteration schemes. The prototype will also guide the development of IQS in the large FEM framework of Rattlesnake/MOOSE, where more complex examples and benchmarks can be tested.  Development in Rattlesnake also spawns the opportunity to test IQS with multiphysics, namely temperature feedback, and time adaptation.