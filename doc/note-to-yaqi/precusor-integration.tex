\documentclass[11pt]{tamurmemo} 
%%%%%%%%%%%%%%%%%%%%%%%%%%%%%%%%%%%%%%%%%%%%%%%%%%%%%%%%%%%%%%%%%%%%

\usepackage{graphics}
\usepackage{float}
\usepackage{subfigure}
\usepackage{graphicx}
\usepackage{amssymb}
\usepackage{amsmath}
\usepackage{amsfonts}
\usepackage{amssymb}
\usepackage{amstext}
\usepackage{amsbsy}
\usepackage{xspace}

%%%%%%%%%%%%%%%%%%%%%%%%%%%%%%%%%%%%%%%%%%%%%%%%%%%%%%%%%%%%%%%%%%%%
% operators
\renewcommand{\div}{\vec{\nabla}\! \cdot \!}
\newcommand{\grad}{\vec{\nabla}}
% latex shortcuts
\newcommand{\bea}{\begin{eqnarray}}
\newcommand{\eea}{\end{eqnarray}}
\newcommand{\be}{\begin{equation}}
\newcommand{\ee}{\end{equation}}
\newcommand{\bal}{\begin{align}}
\newcommand{\eali}{\end{align}}
\newcommand{\bi}{\begin{itemize}}
\newcommand{\ei}{\end{itemize}}
\newcommand{\ben}{\begin{enumerate}}
\newcommand{\een}{\end{enumerate}}
% DGFEM commands
\newcommand{\jmp}[1]{[\![#1]\!]}                     % jump
\newcommand{\mvl}[1]{\{\!\!\{#1\}\!\!\}}             % mean value
\newcommand{\kef}{\ensuremath{k_{\textit{eff}}}}
%\newcommand{\keff}{{\text{k}$_\textit{eff}$}\xspace}
\newcommand{\keff}{\kef\xspace}
% shortcut for domain notation
%\newcommand{\D}{\mathcal{D}}
% vector shortcuts
\newcommand{\vo}{\vec{\Omega}}
\newcommand{\vr}{\vec{r}}
\newcommand{\vn}{\vec{n}}
\newcommand{\vnk}{\vec{\mathbf{n}}}
\newcommand{\vj}{\vec{J}}
% extra space
\newcommand{\qq}{\quad\quad}
% common reference commands
\newcommand{\eqt}[1]{Eq.~(\ref{#1})}                     % equation
\newcommand{\fig}[1]{Fig.~\ref{#1}}                      % figure
\newcommand{\tbl}[1]{Table~\ref{#1}}                     % table

\newcommand{\ud}{\,\mathrm{d}}
\newcommand{\mt}[1]{\marginpar{\tiny #1}}
\newcommand{\D}{\ensuremath{\mathcal{D}}}

%%%%%%%%%%%%%%%%%%%%%%%%%%%%%%%%%%%%%%%%%%%%%%%%%%%%%%%%%%%%%%%%%%%%%
%
%   BEGIN DOCUMENT
%
%%%%%%%%%%%%%%%%%%%%%%%%%%%%%%%%%%%%%%%%%%%%%%%%%%%%%%%%%%%%%%%%%%%%%
\begin{document}
%%%%%%%%%%%%%%%%%%%%%%%%%%%%%%%%%%%%%%%%%%%%%%%%%%%%%%%%%%%%%%%%%%%%%

%%---------------------------------------------------------------------------%%
%% OPTIONS FOR NOTE
%%---------------------------------------------------------------------------%%

\toms{Distribution}
\subject{Precursor Evaluation}

%-------NO CHANGES
\collegename{Dwight Look College of Engineering}
\deptname{Department of Nuclear Engineering}
\fromms{Zachary\ M.\ Prince, Jean C. Ragusa}
\originator{zmp,jcr}
\typist{zmp,jcr}
\date{\today}
%-------NO CHANGES

%-------OPTIONS
%\reference{NPB Star Reimbursable Project}
%\thru{}
%\enc{list}      
%\attachments{list}
\cy{File}
%\encas
%\attachmentas
%\attachmentsas 
%-------OPTIONS

%%---------------------------------------------------------------------------%%
%% DISTRIBUTION LIST
%%---------------------------------------------------------------------------%%

\distribution {
Yaqi Wang, INL, {\em yaqi.wang@inl.gov}
%,\\
%Jean Ragusa, TAMU, {\em jean.ragusa@tamu.edu}
}

%%---------------------------------------------------------------------------%%
%% BEGIN NOTE
%%---------------------------------------------------------------------------%%

\opening

%%---------------------------------------------------------------------------%%
\section{Precursor Solve for Shape Evaluation}
%%---------------------------------------------------------------------------%%

This document presents two different time-integration methods to solve coupled IQS shape + precursor equations, recalled below
using, for simplicity, a single neutron group and a single precursor group.

\be
\frac{1}{v}\frac{\partial\varphi}{\partial t}=\nu\Sigma_f(1-\beta)\varphi-\left(-\div D \grad + \Sigma_a + \frac{1}{v}\frac{1}{p}\frac{dp}{dt}\right)\varphi+\frac{1}{p}\lambda C 
\ee
\be
\frac{dC}{dt} = \beta\nu \Sigma_f \varphi p - \lambda C
\ee

First, we note that we could keep this system of two time-dependent equations and solve it as a coupled system. 
However, this is unnecessary and a memory expensive endeavor because the precursor equation is only an ODE and not a PDE. 
Instead, one may discretize in time the shape equation, which typically requires the knowledge of the
precursor concentrations at the end of the time step. This precursor value is taken from the solution, numerical or anlaytical,
of the precursors ODE. This document will discuss two techniques for solving the precursor equation.  First is a time discretization method that is currently being implemented in YAK.  The second is a analytical integration of the precursors, the latter method has proven to be more beneficial for IQS convergence.


%%---------------------------------------------------------------------------%%
\section{Time Discretization using the Theta Method}
%%---------------------------------------------------------------------------%%

A fairly simple way to evaluate the precursor equation is to employ the $\theta$-scheme ($0\le\theta\le1$,
explicit when $\theta=0$, implicit when $\theta=1$, and Crank-Nicholson when $\theta=1/2$).
Generally, if there is a function $u$ whose governing equation is $\frac{du}{dt}=f(u,t)$,
then the $\theta$-discretization is

\be
\frac{u^{n+1}-u^n}{\Delta t}=(1-\theta)f(u^n,t) + \theta f(u^{n+1},t) \,.
\ee

Applying this to the precursor equation:

\be
\frac{C^{n+1}-C^n}{\Delta t}=(1-\theta)\beta S_f^np^n-(1-\theta)\lambda C^n + \theta\beta S_f^{n+1}p^{n+1}-\theta\lambda C^{n+1}
\ee

Where $S_f$ is the fission source equivalent for shape:

\be
S_f^n=(\nu\Sigma_f)^n\varphi^n
\ee

Rearranging to solve for the precursor at the end of the time step yields

\be
C^{n+1} = \frac{1-(1-\theta)\Delta t\lambda}{1+\theta\Delta t\lambda}C^n + \frac{(1-\theta)\Delta t \beta}{1+\theta\Delta t\lambda}S_f^n p^n +  \frac{\theta\Delta t \beta}{1+\theta\Delta t\lambda}S_f^{n+1} p^{n+1}
\ee
Reporting this value of $C^{n+1}$, one can solve for the shape $\varphi^{n+1}$ as a function of $\varphi^n$ and $C^n$
(and $p^n$, $p^{n+1}$, $dp/dt|_n$ and  $dp/dt|_{n+1}$).
Once $\varphi^{n+1}$ has been determined, $C^{n+1}$ is updated. YAK currently implements both implicit and Crank-Nicholson as options for precursor evaluation.

\bigskip


%%---------------------------------------------------------------------------%%
\section{Analytical Integration}
%%---------------------------------------------------------------------------%%

Through prototyping, it has been found that neither implicit nor Crank-Nicholson time discretization of precursors are preferable methods for solving the shape equation in IQS.  It has been found that these discretizations result in a lack of convergence of the shape over the IQS iteration process.  In order to remedy the error, a analytical representation of the precursors was implemented in the prototype and the shape solution was able to converge (the normalization constant of the IQS method can be preserved to $10^{-10}$ while the theta-scheme only allowed convergence in the normalization factor to about $10^{-3}$).  The following section shows how this method was implemented in the prototype and the desired implementation for YAK.

Using an exponential operator, the precursor equation can be analytically solved for:

\be
\int_{t_n}^{t_{n+1}} C(t')e^{\lambda t'} dt' = \int_{t_n}^{t_{n+1}} \beta(t') S_f(t') p(t')e^{\lambda t'}dt'
\ee

yielding

\be
C^{n+1} =  C^n e^{-\lambda (t_{n+1} - t_n) }  + \int_{t_n}^{t_{n+1}} \beta(t') S_f(t') p(t')e^{-\lambda (t_{n+1}-t')}dt'
\ee

Because $\beta$ and $\nu\Sigma_f\varphi$ being integrated are not known continuously over the time step, they can be interpolated linearly over the macro step.  Such that:

\be
h(t) = \frac{t_{n+1}-t}{t_{n+1}-t_n}h^n  + \frac{t-t_n}{t_{n+1}-t_n}h^{n+1}  \quad t_n \le t \le t_{n+1}
\ee
% We let $\ell_n(t) = \frac{t_{n+1}-t}{t_{n+1}-t_n}$ and  $\ell_{n+1}(t) = \frac{t-t_n}{t_{n+1}-t_n}$ for simplicity. 

However, for the PRKE solve, we do have a very accurate representation of $p(t')$ over the time interval $[t_n,t_{n+1}]$.

Finally, we have the final expression for the analytical value for $C^{n+1}$:

\be
C^{n+1} = C^n e^{-\lambda \Delta t} 
+ \left(a_3\beta^{n+1}+a_2\beta^n\right)S_f^{n+1}
+ \left(a_2\beta^{n+1}+a_1\beta^n\right)S_f^n 
\ee

Where the integration coefficients are defined as:

\begin{align}
&a_1 = \int_{t_n}^{t_{n+1}}\left(\frac{t_{n+1}-t'}{\Delta t}\right)^2p(t')e^{-\lambda(t_{n+1}-t')}dt' \\
&a_2= \int_{t_n}^{t_{n+1}}\frac{(t'-t_n)(t_{n+1}-t')}{(\Delta t)^2}p(t')e^{-\lambda(t_{n+1}-t')}dt' \\
&a_3 = \int_{t_n}^{t_{n+1}}\left(\frac{t'-t_n}{\Delta t}\right)^2p(t')e^{-\lambda(t_{n+1}-t')}dt'
\end{align}

The amplitude $(p)$ is included in the integration coefficient because it has been highly accurately calculated in the micro step scheme, so a piecewise interpolation between those points can be done to maximize accuracy.  

Our prototype code uses Matlab software to interpolate the amplitude between micro steps and a quadrature integration for the coefficients.  So the challenge for YAK is to replicate this procedure: passing the amplitude vector to the DNP auxkernel, interpolating it, and integrating the coefficients.  

\underline{``Note-1'':} Applying this procedure to the fission source (assume a linear in time variation for 
$\beta\nu\Sigma_f(t')\varphi(t')$) is simple. We just need to agree what to do with beta. If beta is independent of time, the equation reduces to:

\be
C^{n+1} = C^n e^{-\lambda \Delta t} + \left(\hat{a}_2 S_f^{n+1}+\hat{a}_1 S_f^n\right)\beta
\ee

\begin{align}
&\hat{a}_1= \int_{t_n}^{t_{n+1}}\frac{t_{n+1}-t'}{\Delta t}p(t')e^{-\lambda(t_{n+1}-t')}dt' \\
&\hat{a}_2 = \int_{t_n}^{t_{n+1}}\frac{t'-t_n}{\Delta t}p(t')e^{-\lambda(t_{n+1}-t')}dt'
\end{align}


\underline{``Note-2'':} We can apply this to the precursor spatial representations as done in YAK, even though
we think the proper way to handle their spatial discretization is the one we proposed in our previous note.


%%%%%%%%%%%%%%%%%%%%%%%%%%%%%%%%%%%%%%%%%%%%%%%%%%%%%%%%%%%%%%%%%%%%%%%%%%%%%%%%%%%%%%%%%%%%%%%%
\newpage
\closing
\end{document}
%%%%%%%%%%%%%%%%%%%%%%%%%%%%%%%%%%%%%%%%%%%%%%%%%%%%%%%%%%%%%%%%%%%%%%%%%%%%%%%%%%%%%%%%%%%%%%%%
