\documentclass[11pt]{tamurmemo} 
%%%%%%%%%%%%%%%%%%%%%%%%%%%%%%%%%%%%%%%%%%%%%%%%%%%%%%%%%%%%%%%%%%%%

\usepackage{graphics}
\usepackage{float}
\usepackage{subfigure}
\usepackage{graphicx}
\usepackage{amssymb}
\usepackage{amsmath}
\usepackage{amsfonts}
\usepackage{amssymb}
\usepackage{amstext}
\usepackage{amsbsy}
\usepackage{xspace}

%%%%%%%%%%%%%%%%%%%%%%%%%%%%%%%%%%%%%%%%%%%%%%%%%%%%%%%%%%%%%%%%%%%%
% operators
\renewcommand{\div}{\vec{\nabla}\! \cdot \!}
\newcommand{\grad}{\vec{\nabla}}
% latex shortcuts
\newcommand{\bea}{\begin{eqnarray}}
\newcommand{\eea}{\end{eqnarray}}
\newcommand{\be}{\begin{equation}}
\newcommand{\ee}{\end{equation}}
\newcommand{\bal}{\begin{align}}
\newcommand{\eali}{\end{align}}
\newcommand{\bi}{\begin{itemize}}
\newcommand{\ei}{\end{itemize}}
\newcommand{\ben}{\begin{enumerate}}
\newcommand{\een}{\end{enumerate}}
% DGFEM commands
\newcommand{\jmp}[1]{[\![#1]\!]}                     % jump
\newcommand{\mvl}[1]{\{\!\!\{#1\}\!\!\}}             % mean value
\newcommand{\kef}{\ensuremath{k_{\textit{eff}}}}
%\newcommand{\keff}{{\text{k}$_\textit{eff}$}\xspace}
\newcommand{\keff}{\kef\xspace}
% shortcut for domain notation
%\newcommand{\D}{\mathcal{D}}
% vector shortcuts
\newcommand{\vo}{\vec{\Omega}}
\newcommand{\vr}{\vec{r}}
\newcommand{\vn}{\vec{n}}
\newcommand{\vnk}{\vec{\mathbf{n}}}
\newcommand{\vj}{\vec{J}}
% extra space
\newcommand{\qq}{\quad\quad}
% common reference commands
\newcommand{\eqt}[1]{Eq.~(\ref{#1})}                     % equation
\newcommand{\fig}[1]{Fig.~\ref{#1}}                      % figure
\newcommand{\tbl}[1]{Table~\ref{#1}}                     % table

\newcommand{\ud}{\,\mathrm{d}}
\newcommand{\mt}[1]{\marginpar{\tiny #1}}
\newcommand{\D}{\ensuremath{\mathcal{D}}}

%%%%%%%%%%%%%%%%%%%%%%%%%%%%%%%%%%%%%%%%%%%%%%%%%%%%%%%%%%%%%%%%%%%%%
%
%   BEGIN DOCUMENT
%
%%%%%%%%%%%%%%%%%%%%%%%%%%%%%%%%%%%%%%%%%%%%%%%%%%%%%%%%%%%%%%%%%%%%%
\begin{document}
%%%%%%%%%%%%%%%%%%%%%%%%%%%%%%%%%%%%%%%%%%%%%%%%%%%%%%%%%%%%%%%%%%%%%

%%---------------------------------------------------------------------------%%
%% OPTIONS FOR NOTE
%%---------------------------------------------------------------------------%%

\toms{Distribution}
\subject{Precursors spatial treatment}

%-------NO CHANGES
\collegename{Dwight Look College of Engineering}
\deptname{Department of Nuclear Engineering}
\fromms{J.\ C.\ Ragusa}
\originator{jcr}
\typist{jcr}
\date{\today}
%-------NO CHANGES

%-------OPTIONS
%\reference{NPB Star Reimbursable Project}
%\thru{}
%\enc{list}      
%\attachments{list}
\cy{File}
%\encas
%\attachmentas
%\attachmentsas 
%-------OPTIONS

%%---------------------------------------------------------------------------%%
%% DISTRIBUTION LIST
%%---------------------------------------------------------------------------%%

\distribution {
Yaqi Wang, INL, {\em yaqi.wang@inl.gov},\\
Zach Prince, TAMU, {\em zachmprince@tamu.edu}
}

%%---------------------------------------------------------------------------%%
%% BEGIN NOTE
%%---------------------------------------------------------------------------%%

\opening

%%---------------------------------------------------------------------------%%
\section{Precursors spatial treatment}
%%---------------------------------------------------------------------------%%

There is no need to expand the precursors as FEM solutions. Let us look at the term
appearing in the flux equation
\be
\ldots = \ldots + \lambda \mathcal{C}(\vr)
\ee
When the flux equation is tested with $\varphi_i(\vr)$ and integrate over the whole computational domain, we obtain, for the delayed neutron source,
\be
\int \lambda \mathcal{C}(\vr) \varphi_i(\vr) = \lambda C_i
\ee
where the unknown $C_i$ is
\be
C_i = \int \mathcal{C}(\vr) \varphi_i(\vr)
\ee

\bigskip

Now, let us take a look at the precursors equation:
\be
\frac{d\mathcal{C}}{dt} = - \lambda \mathcal{C}  + \beta \nu\Sigma_f \Phi
\ee

We only test this equation by $\varphi_i$ and integrate over the whole computational domain (no need to expand $\mathcal{C}$ as a FEM solution)
\be
\frac{dC_i}{dt} = - \lambda C_i  + \int \left( \beta(\vr) \nu\Sigma_f(\vr) \Phi(\vr) \varphi_i(\vr) \right)
\ee
what is $\int \left( \beta \nu\Sigma_f \Phi \varphi_i\right)$ ? It is the $i$-th row of the
$\underline{\underline{M}} \,  \underline{\Phi}$  matrix-vector product, where the entries of matrix $\underline{\underline{M}} $ are
\be
M_{ij} = \int \beta(\vr) \nu\Sigma_f(\vr) \varphi_i(\vr) \varphi_j(\vr)
\ee
and the entries of $\underline{\Phi}$ are the flux nodal values: $\Phi(\vr) = \sum_j \Phi_j \varphi_j(\vr)$.

\bigskip
So the initial values should be computed as
\be
\begin{bmatrix}
C_1 \\
\ldots
\\
C_i \\
\ldots \\
C_n
\end{bmatrix} 
= \frac{1}{\lambda} \underline{\underline{M}} \,  \underline{\Phi}
\ee
If you want the concentration at a quadrature point, you say
\be
\boxed{
C(\vr_q) = \sum_j C_i \varphi_j(\vr_q)
}
\ee
\bigskip
In Yak, the way the precursors are obtained at a quadrature point is as follows
\be
\frac{\beta(\vr_q)}{\lambda(\vr_q)} \times Sf_q
\ee
where the fission source is expanded as
\be
Sf(\vr) = \sum_j Sf_j \varphi_j(\vr) \quad \text{with }
Sf_i = \int  \nu\Sigma_f(\vr) \varphi_i(\vr) \Phi(\vr)
\ee
and thus
\be
\boxed{
C(\vr_q) = \frac{\beta(\vr_q)}{\lambda(\vr_q)} \sum_j Sf_j \varphi_j(\vr_q)
}
\ee
It is not the same thing.


%%%%%%%%%%%%%%%%%%%%%%%%%%%%%%%%%%%%%%%%%%%%%%%%%%%%%%%%%%%%%%%%%%%%%%%%%%%%%%%%%%%%%%%%%%%%%%%%
\newpage
\closing
\end{document}
%%%%%%%%%%%%%%%%%%%%%%%%%%%%%%%%%%%%%%%%%%%%%%%%%%%%%%%%%%%%%%%%%%%%%%%%%%%%%%%%%%%%%%%%%%%%%%%%
