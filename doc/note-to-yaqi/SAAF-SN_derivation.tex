\documentclass[11pt]{tamurmemo} 
%%%%%%%%%%%%%%%%%%%%%%%%%%%%%%%%%%%%%%%%%%%%%%%%%%%%%%%%%%%%%%%%%%%%

\usepackage{graphics}
\usepackage{float}
\usepackage{subfigure}
\usepackage{graphicx}
\usepackage{amssymb}
\usepackage{amsmath}
\usepackage{amsfonts}
\usepackage{amssymb}
\usepackage{amstext}
\usepackage{amsbsy}
\usepackage{xspace}
\usepackage{verbatim}
\usepackage{csvsimple}

%%%%%%%%%%%%%%%%%%%%%%%%%%%%%%%%%%%%%%%%%%%%%%%%%%%%%%%%%%%%%%%%%%%%
% operators
\renewcommand{\div}{\vec{\nabla}\! \cdot \!}
\newcommand{\grad}{\vec{\nabla}}
% latex shortcuts
\newcommand{\bea}{\begin{eqnarray}}
\newcommand{\eea}{\end{eqnarray}}
\newcommand{\be}{\begin{equation}}
\newcommand{\ee}{\end{equation}}
\newcommand{\bal}{\begin{align}}
\newcommand{\eali}{\end{align}}
\newcommand{\bi}{\begin{itemize}}
\newcommand{\ei}{\end{itemize}}
\newcommand{\ben}{\begin{enumerate}}
\newcommand{\een}{\end{enumerate}}
% DGFEM commands
\newcommand{\jmp}[1]{[\![#1]\!]}                     % jump
\newcommand{\mvl}[1]{\{\!\!\{#1\}\!\!\}}             % mean value
\newcommand{\kef}{\ensuremath{k_{\textit{eff}}}}
%\newcommand{\keff}{{\text{k}$_\textit{eff}$}\xspace}
\newcommand{\keff}{\kef\xspace}
% shortcut for domain notation
%\newcommand{\D}{\mathcal{D}}
% vector shortcuts
\newcommand{\vo}{\vec{\Omega}}
\newcommand{\vr}{\vec{r}}
\newcommand{\vn}{\vec{n}}
\newcommand{\vnk}{\vec{\mathbf{n}}}
\newcommand{\vj}{\vec{J}}
% extra space
\newcommand{\qq}{\quad\quad}
% common reference commands
\newcommand{\eqt}[1]{Eq.~(\ref{#1})}                     % equation
\newcommand{\fig}[1]{Fig.~\ref{#1}}                      % figure
\newcommand{\tbl}[1]{Table~\ref{#1}}                     % table

\newcommand{\ud}{\,\mathrm{d}}
\newcommand{\mt}[1]{\marginpar{\tiny #1}}
\newcommand{\D}{\ensuremath{\mathcal{D}}}
\allowdisplaybreaks

%%%%%%%%%%%%%%%%%%%%%%%%%%%%%%%%%%%%%%%%%%%%%%%%%%%%%%%%%%%%%%%%%%%%%
%
%   BEGIN DOCUMENT
%
%%%%%%%%%%%%%%%%%%%%%%%%%%%%%%%%%%%%%%%%%%%%%%%%%%%%%%%%%%%%%%%%%%%%%
\begin{document}
%%%%%%%%%%%%%%%%%%%%%%%%%%%%%%%%%%%%%%%%%%%%%%%%%%%%%%%%%%%%%%%%%%%%%

%%---------------------------------------------------------------------------%%
%% OPTIONS FOR NOTE
%%---------------------------------------------------------------------------%%

\toms{Distribution}
\subject{SAAF-$S_n$ Derivation}

%-------NO CHANGES
\collegename{Dwight Look College of Engineering}
\deptname{Department of Nuclear Engineering}
\fromms{Zachary\ M.\ Prince}
\originator{zmp,jcr}
\typist{zmp,jcr}
\date{\today}
%-------NO CHANGES

%-------OPTIONS
%\reference{NPB Star Reimbursable Project}
%\thru{}
%\enc{list}      
%\attachments{list}
\cy{File}
%\encas
%\attachmentas
%\attachmentsas 
%-------OPTIONS

%%---------------------------------------------------------------------------%%
%% DISTRIBUTION LIST
%%---------------------------------------------------------------------------%%

\distribution {
Yaqi Wang, INL, {\em yaqi.wang@inl.gov}
%,\\
%Jean Ragusa, TAMU, {\em jean.ragusa@tamu.edu}
}

%%---------------------------------------------------------------------------%%
%% BEGIN NOTE
%%---------------------------------------------------------------------------%%

\opening

%%---------------------------------------------------------------------------%%
\section{Derivation of IQS with SAAF-$S_n$}
%%---------------------------------------------------------------------------%%

Multigroup $S_n$ equation with delayed neutron precursors:
\begin{align}
\frac{\partial}{\partial t}\left(\frac{\Psi^g}{v^g}\right) =& \frac{\chi_p^g}{4\pi} \sum_{g'=1}^G (1-\beta) \frac{\nu^{g'} \Sigma_f^{g'}}{\keff} \phi^{g'} -  \left( \vo \cdot \grad + \Sigma_t^g \right) \Psi^g  \nonumber \\
&  + \sum_{g'=1}^G\sum_{l=0}^N \frac{2l+1}{4\pi}\sum_{m=-l}^l\Sigma_{s,l}^{g'\to g} \phi^{g'}_{l,m}Y_{l,m} + \sum_{i=1}^I\frac{\chi_{d,i}^g}{4\pi}\lambda_i C_i \ , \quad 1 \le g \le G 
\label{eq:diffusion} \\
\frac{dC_i}{dt} =& \beta_i\sum_{g=1}^G \frac{\nu^{g} \Sigma_f^{g}}{\keff} \phi^{g} - \lambda_i C_i \ , \quad 1 \le i \le I 
\label{eq:prec}
\end{align}

With surface-source and reflecting boundary conditions:
\begin{align}
\Psi^g(\vr_b,\vo) = & \begin{cases} \Psi_{inc}^g(\vr_b,\vo), & \vr_b\in\partial\D_s \\  \Psi^g(\vr_b,\vo_r), & \vr_b\in\partial\D_r \end{cases} \\
 & \text{for } \vo\cdot\vn_b<0
\end{align}

Where $\vn_b$ is the outward unit vector at point $\vr_b$ on the boundary, $\partial\D=\partial\D_s \cup \partial\D_r$, and:
\be
\phi^g_{l,m} = \int_{4\pi}\Psi^gY_{l,m}d^2\Omega
\ee
\be
\Sigma_{s,l}^{g'\to g} = \int_{-1}^1 \Sigma_{s}^{g'\to g} P_l d\mu
\ee
\be
\vo_r = \vo -2(\vo\cdot\vn_b)\vn_b
\ee

Multiplying Equation~\ref{eq:diffusion} by $\Psi^{*g}+\tau^g\vo\cdot\grad\Psi^{*g}$ and Equation~\ref{eq:prec} by $C_i^*=\sum_{g=1}^G\chi_{d,i}^g\phi^{*g}$ for test functions (where $\Psi^{*g}$ is the initial adjoint flux and is independent of time) and integrating in space and angle:
\begin{align}
&\left(\Psi^{*g}+\tau^g\vo\cdot\grad\Psi^{*g},\frac{\partial}{\partial t}\left(\frac{\Psi^g}{v^g}\right)\right)_\D = \left(\left(\Psi^{*g}+\tau^g\vo\cdot\grad\Psi^{*g}\right)\chi_p^g(1-\beta),\sum_{g'=1}^G \frac{\nu^{g'} \Sigma_f^{g'}}{\keff} \phi^{g'}\right)_\D \nonumber \\
& \qq - \left(\Psi^{*g}+\tau^g\vo\cdot\grad \Psi^{*g}, \left(\vo\cdot\grad + \Sigma_t^g\right) \Psi^g\right)_\D \nonumber \\
& \qq + \sum_{g'=1}^G\left(\Psi^{*g}+\tau^g\vo\cdot\grad\Psi^{*g},\sum_{l=0}^N \frac{2l+1}{4\pi}\sum_{m'=-l}^l\Sigma_{s,l}^{g'\to g} \phi^{g'}_{l,m}Y_{l,m}\right)_\D \nonumber \\
& \qq + \sum_{i=1}^I \left(\left(\Psi^{*g}+\tau^g\vo\cdot\grad\Psi^{*g}\right)\chi_{d,i}^g,\lambda_i C_i\right)_\D
\label{eq:tranweak}
\end{align}
\be
\frac{d}{dt}\left(C_i^*,C_i\right)_\D = \left(C_i^* \beta_i,\sum_{g'=1}^G\frac{\nu^{g'} \Sigma_f^{g'}}{\keff}\phi^{g'}\right)_\D -\left(C_i^*,\lambda_i C_i\right)_\D
\ee

Where the parenthetical notation is defined as follows:
\be
\int_\D\int_{4\pi} f(\vr,\vo) g(\vr,\vo) d^3r d^2\Omega = (f,g)_\D
\ee

Usually $\tau^g=1/\Sigma_t^g$, but $\tau^g$ is used as a stabilization parameter for void treatment and is defined by Equation~\ref{eq:tau}.  Where $h$ is the mesh vertex separation distance and $\varsigma$ is the stabilization parameter, taken to be 0.5.
\be
\tau^g = \begin{cases} \frac{1}{\Sigma_t^g} & h\Sigma_t^g\geq\varsigma \\ \frac{h}{\varsigma} & h\Sigma_t^g<\varsigma \end{cases}
\label{eq:tau}
\ee

The second term on the right hand side of Equation~\ref{eq:tranweak} can be expanded as such:
\begin{align}
\left(\Psi^{*g}+\tau^g\vo\cdot\grad \Psi^{*g}, \left(\vo\cdot\grad + \Sigma_t^g\right) \Psi^g\right)_\D &= \left(\tau^g\vo\cdot\grad \Psi^{*g}, \vo\cdot\grad \Psi^g\right)_\D  \nonumber \\
 + \left(\tau^g\vo\cdot\grad \Psi^{*g}, \Sigma_t^g \Psi^g\right)_\D &+ \left(\Psi^{*g}, \vo\cdot\grad \Psi^g\right)_\D + \left(\Psi^{*g}, \Sigma_t^g\Psi^g\right)_\D
\label{eq:expan}
\end{align}

Using integration by parts on the third term of the right hand side of Equation~\ref{eq:expan}:
\be
\int_{4\pi}d^2\Omega\int_\D\Psi^{*g} \vo\cdot\grad \Psi^g d^3r = \int_{4\pi}d^2\Omega\int_\D\grad(\Psi^{*g}\vo\cdot\Psi^g)d^3r - \int_{4\pi}d^2\Omega\int_\D \vo\cdot\grad \Psi^{*g}\Psi^g d^3r
\label{eq:ibp}
\ee

Using Green's theorem on the first term on the right hand side of Equation~\ref{eq:ibp}:
\begin{align}
&\int_{4\pi}d^2\Omega\int_\D\grad(\Psi^{*g}\vo\cdot\Psi^g)d^3r = \int_{4\pi}d^2\Omega\oint_{\partial \D} \vo\cdot\vn_b\Psi^{*g}\Psi^g d^2r \\
&= \oint_{\partial \D}d^2r\int_{\vo\cdot\vn_b>0}d^2\Omega \left|\vo\cdot\vn_b\right| \Psi^{*g}\Psi^g - \oint_{\partial \D}d^2r\int_{\vo\cdot\vn_b<0}d^2\Omega \left|\vo\cdot\vn_b\right| \Psi^{*g}\Psi^g \\
&= \oint_{\partial \D}d^2r\int_{\vo\cdot\vn_b>0}d^2\Omega \left|\vo\cdot\vn_b\right| \Psi^{*g}\Psi^g - \oint_{\partial \D_s}d^2r\int_{\vo\cdot\vn_b <0}d^2\Omega\left|\vo\cdot\vn_b\right| \Psi^{*g}\Psi^g_{inc} \nonumber \\
& - \oint_{\partial \D_r}d^2r\int_{\vo\cdot\vn_b <0}d^2\Omega\left|\vo\cdot\vn_b\right| \Psi^{*g}(\vo_r)\Psi^g(\vo_r)
\label{eq:green}
\end{align}

Applying Equations~\ref{eq:ibp}~and~\ref{eq:green} to Equation~\ref{eq:tranweak}:
\begin{align}
&\left(\Psi^{*g}+\tau^g\vo\cdot\grad\Psi^{*g},\frac{\partial}{\partial t}\left(\frac{\Psi^g}{v^g}\right)\right)_\D = \left(\left(\Psi^{*g}+\tau^g\vo\cdot\grad\Psi^{*g}\right)\chi_p^g(1-\beta),\sum_{g'=1}^G \frac{\nu^{g'} \Sigma_f^{g'}}{\keff} \phi^{g'}\right)_\D \nonumber \\
& \qq - \left(\vo\cdot\grad\Psi^{*g}, \tau^g\vo\cdot\grad\Psi^g - (1 - \tau^g\Sigma_t^g)\Psi^g\right)_\D - \left\langle\Psi^{*g},\Psi^{g}\right\rangle_{\partial \D}^+ + \left\langle\Psi^{*g},\Psi^{g}_{inc}\right\rangle_{\partial \D_s}^- + \left\langle\Psi^{*g},\Psi^{g}\right\rangle_{\partial \D_r}^- \nonumber \\
& \qq - \left(\Psi^{*g},\Sigma_t^g\Psi^g\right)_\D + \sum_{g'=1}^G\left(\Psi^{*g}+\tau^g\vo\cdot\grad\Psi^{*g},\sum_{l=0}^N \frac{2l+1}{4\pi}\sum_{m'=-l}^l\Sigma_{s,l}^{g'\to g} \phi^{g'}_{l,m}Y_{l,m}\right)_\D \nonumber \\
& \qq + \sum_{i=1}^I \left(\left(\Psi^{*g}+\tau^g\vo\cdot\grad\Psi^{*g}\right)\chi_{d,i}^g,\lambda_i C_i\right)_\D
\label{eq:diffweak}
\end{align}

Where the boundary parenthetical notation is defined as:
\be
\oint_{\partial \D}d^2r\int_{\vo\cdot\vn_b >0}d^2\Omega\left|\vo\cdot\vn_b\right|f(\vr,\vo) g(\vr,\vo) = \left\langle f,g \right\rangle_{\partial \D}^+
\ee
\be
\oint_{\partial \D_s}d^2r\int_{\vo\cdot\vn_b <0}d^2\Omega\left|\vo\cdot\vn_b\right|f(\vr_b,\vo) g(\vr_b,\vo) = \left\langle f,g \right\rangle_{\partial \D_s}^-
\ee
\be
\oint_{\partial \D_r}d^2r\int_{\vo\cdot\vn_b <0}d^2\Omega\left|\vo\cdot\vn_b\right|f(\vr_b,\vo_r) g(\vr_b,\vo_r) = \left\langle f,g \right\rangle_{\partial \D_r}^-
\ee

Summing over groups on both sides of Equation~\ref{eq:diffweak}:
\begin{align}
&\sum_{g=1}^G\left(\Psi^{*g}+\tau^g\vo\cdot\grad\Psi^{*g},\frac{\partial}{\partial t}\left(\frac{\Psi^g}{v^g}\right)\right)_\D = \left(\sum_{g=1}^G\left(\Psi^{*g}+\tau^g\vo\cdot\grad\Psi^{*g}\right)\chi_p^g(1-\beta),\sum_{g'=1}^G \frac{\nu^{g'} \Sigma_f^{g'}}{\keff} \phi^{g'}\right)_\D \nonumber \\
& \hspace{-5mm} - \sum_{g=1}^G\left(\vo\cdot\grad\Psi^{*g}, \tau^g\vo\cdot\grad\Psi^g - (1 - \tau^g\Sigma_t^g)\Psi^g\right)_\D - \sum_{g=1}^G\left(\left\langle\Psi^{*g},\Psi^{g}\right\rangle_{\partial \D}^+ - \left\langle\Psi^{*g},\Psi^{g}_{inc}\right\rangle_{\partial \D_s}^- - \left\langle\Psi^{*g},\Psi^{g}\right\rangle_{\partial \D_r}^-\right)  \nonumber \\
& \qq - \sum_{g=1}^G\left(\Psi^{*g},\Sigma_t^g\Psi^g\right)_\D + \sum_{g=1}^G\sum_{g'=1}^G\left(\Psi^{*g}+\tau^g\vo\cdot\grad\Psi^{*g},\sum_{l=0}^N \frac{2l+1}{4\pi}\sum_{m=-l}^l\Sigma_{s,l}^{g'\to g} \phi^{g'}_{l,m}Y_{l,m}\right)_\D \nonumber \\
& \qq + \sum_{i=1}^I \left(\sum_{g=1}^G\left(\Psi^{*g}+\tau^g\vo\cdot\grad\Psi^{*g}\right)\chi_{d,i}^g,\lambda_i C_i\right)_\D
\end{align}

Defining bilinear functions:
\begin{align}
T(\Psi^{*},\Psi) =& \sum_{g=1}^G\left(\Psi^{*g}+\tau^g\vo\cdot\grad\Psi^{*g},\frac{\partial}{\partial t}\left(\frac{\Psi^g}{v^g}\right)\right)_\D \\
C_i(C_i^*,C_i) =& \left(C_i^*,C_i\right)_\D \\
F(\Psi^{*},\Psi) =& \left(\sum_{g=1}^G\left(\Psi^{*g}+\tau^g\vo\cdot\grad\Psi^{*g}\right)\chi_p^g(1-\beta)+\sum_{i=1}^I\chi_{d,i}^g\beta_i\phi^{*g},\sum_{g'=1}^G \frac{\nu^{g'} \Sigma_f^{g'}}{\keff} \phi^{g'}\right)_\D \\
L(\Psi^{*},\Psi) =& \sum_{g=1}^G\left(\vo\cdot\grad\Psi^{*g}, \tau^g\vo\cdot\grad\Psi^g - (1 - \tau^g\Sigma_t^g)\Psi^g\right)_\D + \sum_{g=1}^G\left\langle\Psi^{*g},\Psi^{g}\right\rangle_{\partial \D}^+ \nonumber \\
& \qq - \sum_{g=1}^G\left\langle\Psi^{*g},\Psi^{g}_{inc}\right\rangle_{\partial \D_s}^- - \sum_{g=1}^G\left\langle\Psi^{*g},\Psi^{g}\right\rangle_{\partial \D_r}^- + \sum_{g=1}^G\left(\Psi^{*g},\Sigma_t^g\Psi^g\right)_\D \\
S(\Psi^{*},\Psi) =& \sum_{g=1}^G\sum_{g'=1}^G\left(\Psi^{*g}+\tau^g\vo\cdot\grad\Psi^{*g},\sum_{l=0}^N \frac{2l+1}{4\pi}\sum_{m'=-l}^l\Sigma_{s,l}^{g'\to g} \phi^{g'}_{l,m}Y_{l,m}\right)_\D \\
F_{d,i}(\Psi^*,\Psi) =& \left(\beta_i\sum_{g=1}^G\chi_{d,i}^g\phi^{*g},\sum_{g'=1}^G\frac{\nu^{g'} \Sigma_f^{g'}}{\keff}\phi^{g'}\right)_\D \\
S_{d,i}(\Psi^*,\Psi) =& \left(\sum_{g=1}^G\chi_{d,i}^g\phi^{*g},\lambda_i C_i\right)_\D \\
P_i(C^*_i,\Psi) = & \left(C^*_i\beta_i,\sum_{g'=1}^G\frac{\nu^{g'} \Sigma_f^{g'}}{\keff}\phi^{g'}\right)_\D \\
D_i(C^*_i,C_i) =& \left(C_i^*,\lambda_i C_i\right)_\D
\end{align}

Rewriting the flux and precursor equations:
\be
T(\Psi^{*},\Psi) = F(\Psi^{*},\Psi) - \sum_{i=1}^I F_{d,i}(\Psi^{*},\Psi) - L(\Psi^{*},\Psi) + S(\Psi^{*},\Psi) + \sum_{i=1}^I S_{d,i}(\Psi^*,C_i)
\ee
\be
\frac{d}{dt}C_i(C_i^*,C_i) = P_i(C_i^*,\Psi) - D_i(C_i^*,C_i)
\ee

Doing IQS factorization ($\Psi^g(\vr,\vo,t) = p(t)\psi^g(\vr,\vo,t)$):
\begin{align}
T(\Psi^{*},\psi) &= F(\Psi^{*},\psi) - \sum_{i=1}^I F_{d,i}(\Psi^*,\psi) - L(\Psi^{*},\psi) - \frac{1}{p}\frac{dp}{dt}\sum_{g=1}^G\left(\Psi^{*g}+\tau^g\vo\cdot\grad\Psi^{*g},\frac{\psi^g}{v^g}\right)_\D \nonumber \\
& + S(\Psi^{*},\psi) + \frac{1}{p}\sum_{i=1}^I S_{d,i}(\Psi^*,C_i)
\label{eq:shape1}
\end{align}
\be
\frac{d}{dt}C_i(C_i^*,C_i) = P_i(C_i^*,\psi)p - D_i(C_i^*,C_i)
\label{eq:cshape1}
\ee

Equations~\ref{eq:shape1}~and~\ref{eq:cshape1} are the equations used for calculation of IQS shape.  In order to continue to the PRKE derivation, time derivative term needs to be taken out of $T(\Psi^{*},\psi)$:

\be
T(\Psi^{*},\psi) = \frac{d}{dt}\sum_{g=1}^G\left(\Psi^{*g},\frac{\psi^g}{v^g}\right)_\D + \sum_{g=1}^G\left(\tau^g\vo\cdot\grad\Psi^{*g},\frac{\partial}{\partial t}\left(\frac{\psi^g}{v^g}\right)\right)_\D
\ee

Defining new terms in bilinear form:
\begin{align}
I(\Psi^{*},\psi) &= \sum_{g=1}^G\left(\Psi^{*g},\frac{\psi^g}{v^g}\right)_\D \\
Z(\Psi^{*},\psi) &= \sum_{g=1}^G\left(\tau^g\vo\cdot\grad\Psi^{*g},\frac{\partial}{\partial t}\left(\frac{\psi^g}{v^g}\right)\right)_\D
\end{align}

Defining PRKE parameters:
\be
\rho = \frac{F(\Psi^{*},\psi) - L(\Psi^{*},\psi) + S(\Psi^{*},\psi)}{F(\Psi^{*},\psi)}
\ee
\be
\bar{\beta} = \sum_{i=1}^I\bar{\beta}_i = \sum_{i=1}^I\frac{F_{d,i}(\Psi^*,\psi)}{F(\Psi^{*},\psi)} = \sum_{i=1}^I\frac{P_i(C_i^*,\psi)}{F(\Psi^{*},\psi)}
\ee
\be
\Lambda = \frac{I(\Psi^{*},\psi)+Z(\Psi^{*},\psi)}{F(\Psi^{*},\psi)}
\ee
\be
\bar{\lambda}_i = \frac{S_{d,i}(\Psi^*,C_i)}{C_i(C_i^*,C_i)} = \frac{D_i(C_i^*,C_i)}{C_i(C_i^*,C_i)}
\ee
\be
\xi_i = \frac{C_i(C_i^*,C_i)}{I(\Psi^{*},\psi)+Z(\Psi^{*},\psi)}
\ee

New term for SAAF-SN:
\be 
\zeta = \frac{Z(\Psi^*,\psi)}{F(\Psi^{*},\psi)}
\ee

Implementing the parameters:
\be
\frac{dp}{dt}=\left[\frac{\rho-\bar{\beta}-\zeta}{\Lambda}\right]p + \sum_{i=1}^I\bar{\lambda}_i\xi_i - p\frac{1}{I(\Psi^{*},\psi)+Z(\Psi^{*},\psi)}\frac{d}{dt} I(\Psi^{*},\psi)
\ee
\be
\frac{d\xi_i}{dt}=\frac{\bar{\beta}_i}{\Lambda}p - \bar{\lambda}_i\xi_i \quad 1 \le i \le I 
\ee

Assuming $\frac{d}{dt} I(\Psi^{*},\psi) = 0$:
\be
\frac{dp}{dt}=\left[\frac{\rho-\bar{\beta}-\zeta}{\Lambda}\right]p+\sum_{i=1}^I\bar{\lambda}_i\xi_i
\label{eq:p}
\ee
\be
\frac{d\xi_i}{dt}=\frac{\bar{\beta}_i}{\Lambda}p - \bar{\lambda}_i\xi_i\quad 1 \le i \le I 
\label{eq:xi}
\ee

%%---------------------------------------------------------------------------%%
\section{Rattlesnake Changes}
%%---------------------------------------------------------------------------%%

It seems that the implementation of SAAF-$S_n$ is less trivial than previously expected.  First, the test function in SAAF is more complicated than just the adjoint flux, as is in diffusion.  So the post-processors that calculate the PRKE parameters need to be rewritten to reflect this.  Second, the test function is also time dependent, so the time derivative on the forward flux cannot be taken out immediately and applied to the entire integral.  As a result, the $\zeta$ term needs to be added to the PRKE within the the IQS user object and a post-processor that evaluates the numerator needs to be added. It might be easier in the end to create a general user object that does all of this in one file.




%%%%%%%%%%%%%%%%%%%%%%%%%%%%%%%%%%%%%%%%%%%%%%%%%%%%%%%%%%%%%%%%%%%%%%%%%%%%%%%%%%%%%%%%%%%%%%%%
\newpage
\closing
\end{document}
%%%%%%%%%%%%%%%%%%%%%%%%%%%%%%%%%%%%%%%%%%%%%%%%%%%%%%%%%%%%%%%%%%%%%%%%%%%%%%%%%%%%%%%%%%%%%%%%
